	%!TEX root = ./main.tex

%\documentclass[11pt, a4paper, titlepage, slovak]{book}
\documentclass[12pt, a4paper, titlepage, slovak]{book}

%Includes all neccessary packets and contains some useful commands
%!TEX root = ./main.tex

%----------------------------------------------------------------------------------
%Packages and parameters
%----------------------------------------------------------------------------------

%inputencoding for the mac
%\documentclass[slovak]{article}
\usepackage[T1]{fontenc}
\usepackage[utf8]{inputenc}
\usepackage{babel}

% =========== OLD SETTINGS =========== %
%\usepackage[utf8]{inputenc}
%\usepackage[slovak,english]{babel}
% ================================== %

%packages for mathematical symbols
\usepackage{latexsym}
\usepackage{amsmath}
\usepackage{mathtools}
\usepackage{amssymb}

\usepackage{pdfpages}

% removes 'chapter' word from titles and apply own style
\usepackage{titlesec}
\titleformat{\chapter}{\normalfont\huge\bf}{\thechapter.}{20pt}{\huge\bf}

%table design
\usepackage{booktabs}

%grahics package
\usepackage{color,graphicx}

%path to your image folder
\graphicspath{{images/}}
\usepackage{float} % package for floating images in text

%do not indent at new paragraphs but add a vertical offset
\usepackage{noindent}

%change standard fonts to Palatino and Helvetica
\usepackage{helvet}

%bibliography settings
\usepackage{natbib}
\bibliographystyle{apalike} %harvad style
%\bibliographystyle{ieeepes}

%source code listings
\usepackage{color}
\usepackage{xcolor}
\usepackage{listings}
\usepackage{caption}
\DeclareCaptionFont{white}{\color{white}}
\DeclareCaptionFormat{listing}{\colorbox{gray}{\parbox{\textwidth}{#1#2#3}}}
\captionsetup[lstlisting]{format=listing,labelfont=white,textfont=white}
\renewcommand{\lstlistingname}{Algoritmus}

%highlits
\usepackage{framed}
\definecolor{shadecolor}{rgb}{0.75,0.75,0.75}

%citation commands
\newcommand{\fullcite}{\citep} %for "Author [1980]"
\renewcommand{\citeyear}{\citeyearpar} %for "[1980]"

\usepackage{setspace}
%\doublespacing
\onehalfspacing

\usepackage[]{algorithm2e}

% change images numbering to per subsection
%\usepackage{chngcntr}
%\counterwithin{figure}{section}
%\counterwithin{table}{section}


%package for control structures
\usepackage{ifthen}

%marginpar hack --- moves margin notes to correct position
\usepackage{mparhack}

%make readable references
\usepackage[pdftex,plainpages=false,pdfpagelabels]{hyperref}


%---------------------<Layout in the style of "A Pattern Approach to Interaction Design>---------------------------

% Change page headers and footers:
\usepackage{fancyhdr}
\pagestyle{fancy}
\fancyhf{}
\fancyhead[RE]{\slshape \nouppercase{\leftmark}}    % Even page header: "page   chapter"
\fancyhead[LO]{\slshape \nouppercase{\rightmark}}   % Odd  page header: "section   page"
\fancyhead[RO,LE]{\bfseries \thepage} 
\renewcommand{\headrulewidth}{1pt}    % Underline headers
\renewcommand{\footrulewidth}{0pt}    

\fancypagestyle{plain}{               % No chapter+section on chapter start pages
\fancyhf{}
\fancyhead[RO,LE]{\bfseries \thepage}
\renewcommand{\headrulewidth}{1pt}
\renewcommand{\footrulewidth}{0pt}
}

% Left headings: "1  INTRODUCTION"
\renewcommand{\chaptermark}[1]{%
\markboth{\thechapter\ \ \ \ #1}{}}

% Right headings: "1.1  Basics"
\renewcommand{\sectionmark}[1]{%
\markright{\thesection\ \ \ \ #1}{}}


% ZMENA TABULKA ======
\renewcommand{\tablename}{Tabuľka}

% some Fancyhdr problem...
\addtolength{\headheight}{2pt} % To avoid overfull vboxes from fancyhdr

%creating a better way to change the layout for the abstract pages
\usepackage{geometry}

\titlespacing*{\chapter}{0pt}{0pt}{25pt}

\ifthenelse{\lengthtest{\paperheight=250mm}}%
{% -----------------B5 Layout-----------------
% Page layout

%\pdfpageheight250mm
%\pdfpagewidth176mm
\geometry{	b5paper,
			top = 27mm,
			footskip = 10mm,
			inner = 19mm,
			outer = 39mm,
			textheight = 175mm,
			textwidth = 84mm,
			marginparsep = 3mm,
			marginparwidth = 32mm
}
\savegeometry{myText}
% -----------------/ B5 Layout-----------------
}%
{% -----------------A4 Layout-----------------
% Page layout
\geometry{	a4paper,
			twoside,
			includemp,
			includehead,
			top = 20mm,
			headsep = 10mm,
			bindingoffset = 0mm,
			inner = 40mm,
			outer = 30mm,
			bottom = 30mm,
			marginparsep = 0mm,
			marginparwidth = 0mm
}
\savegeometry{myText}
% -----------------/ A4 Layout-----------------
}
% Abstract layout
\geometry{	marginparsep = 0mm,
			marginparwidth = 0mm
}
\savegeometry{myAbstract}
\loadgeometry{myText}


\newlength{\fullwidth} % Width of text plus margin notes
\setlength{\fullwidth}{\textwidth}
\addtolength{\fullwidth}{\marginparsep}
\addtolength{\fullwidth}{\marginparwidth}

\setlength{\headwidth}{\fullwidth} % Header stretches over margin notes


%---------------------</Layout in the style of "A Pattern Approach to Interaction Design>---------------------------


%wird fŸr die flŠchendeckende Ausgabe der Titelseite benštigt
%needed for the full-face titlepage
\usepackage{eso-pic}

%index verwenden
%make an index
\usepackage{makeidx}
\makeindex

%Index Formatierungshilfen
%formatting helpers for the index
\newcommand{\uu}[1]{\underline{#1}}
\newcommand{\ii}[1]{\textit{#1}}

%neue Definition der Index Umgebung
%redesign of the index
\renewenvironment{theindex}{%
  \vspace*{50pt}%
  {\Huge\bfseries\indexname}\par%
  \vspace*{40pt}%
  \setlength{\parskip}{0pt}%
  \setlength{\parindent}{0pt}%
  \small%
  \renewcommand{\item}{\par{}}%
  \renewcommand{\subitem}{\par\hspace{2em}- }%
}%
{}

%Maximale Gliederungstiefe, die noch ins Inhaltsverzeichnis aufgenommen wird
%maximum depth for the table of contents
\setcounter{tocdepth}{3}

%Vorschlag fŸr ein schšnes Farbschema
%Set of colors which look nice together
\usepackage{color}
\definecolor{orange_light}{rgb}{1,0.8,0.4}
\definecolor{orange_med}{rgb}{0.753,0.62,0.373}
\definecolor{orange_dark}{rgb}{0.506,0.412,0.251}

\definecolor{green_light}{rgb}{0.8,1,0.4}
\definecolor{green_med}{rgb}{0.635,0.745,0.376}
\definecolor{green_dark}{rgb}{0.435,0.498,0.255}

\definecolor{blue_light}{rgb}{0.4,0.8,1}
\definecolor{blue_med}{rgb}{0.365,0.624,0.749}
\definecolor{blue_dark}{rgb}{0.251,0.42,0.502}

\definecolor{pink_light}{rgb}{1,0.435,0.812}
\definecolor{pink_med}{rgb}{0.745,0.38,0.62}
\definecolor{pink_dark}{rgb}{0.498,0.255,0.416}

\definecolor{yellow_light}{rgb}{1,1,0.4}
\definecolor{yellow_med}{rgb}{0.757,0.745,0.373}
\definecolor{yellow_dark}{rgb}{0.506,0.49,0.251}

%blau (fŸr URLs)
%blue (for URLs)
\definecolor{blue}{rgb}{0,0,1}

%notwendig fŸr die korrekte Erkennung, auf welcher Seite sich eine Abbildung befindet.
%we need this to determine if a figure is on an odd or even page
\usepackage{chngpage}

%Hiermit kšnnen die Abbildungslegenden frei gestaltet werden
%we need this to redesign the captions
\usepackage[font=normalsize,labelfont=bf]{caption}

%Abbildungen kommen auf eine eigene Seite, wenn sie mehr als 85% des Platzes
%auf einer Seite einnehmen
%if a figure takes more than 85% of a page it will be typeset on a separate page
\renewcommand{\floatpagefraction}{0.85}

%Benštigt um gro§e Abbildungen gedreht auf eine Seite zu setzen
%we need this to rotate big figures
\usepackage[figuresright]{rotating}

%Verschiedene LŠngenma§e fŸr Textboxen
%dimensions for textboxes
\newlength{\myDefBoxWidth}
\setlength{\myDefBoxWidth}{\textwidth}
\addtolength{\myDefBoxWidth}{-4mm}
\newlength{\myBigDefBoxWidth}
\setlength{\myBigDefBoxWidth}{\fullwidth}
\addtolength{\myBigDefBoxWidth}{-4mm}

%Formathilfen fŸr MatLab Code (wer's braucht...)
%pre-defined matlab code formats
\usepackage{alltt}
\definecolor{string}{rgb}{0.7,0.0,0.0}
\definecolor{comment}{rgb}{0.13,0.54,0.13}
\definecolor{keyword}{rgb}{0.0,0.0,1.0}



%-----------------------------------------------------------------------------------------------------------------------------------------
% neue Befehle
% new commands
%-----------------------------------------------------------------------------------------------------------------------------------------

%----------------------------------------------------------------------------------

% \myBigFigure	[ LABEL_PREFIX (optional) ]
%				{ FILENAME (without extension) }
%				{ CAPTION TEXT }
%				{ SHORT VERSION OF CAPTION TEXT }
%
%Bild wird in kompletter Breite gesetzt
%picture using full width of the page
\newcommand{\myBigFigure}[4][image]
{%
\begin{figure}[ht!bp]%
%\begin{figure}[H]%
	\renewcommand{\figurename}{Obrázok}
	\checkoddpage%
	\ifcpoddpage%
		%nothing
	\else
		\hspace{-\marginparsep}\hspace{-\marginparwidth}%
	\fi
	%use minipage to center the label beneath the figure
	\begin{minipage}{\fullwidth}%
		\includegraphics[width= \fullwidth]{#2}%
		\caption[#4]{#3}%
		\label{#1_#2}%
	\end{minipage}%
\end{figure}%
}


%----------------------------------------------------------------------------------
% \myFrameBigFigure	[ LABEL_PREFIX (optional) ]
%					{ FILENAME (without extension) }
%					{ CAPTION TEXT }
%					{ SHORT VERSION OF CAPTION TEXT }
%
%Bild wird in kompletter Breite gesetzt und eingerahmt
%picture with frame using the full width of the page
\newcommand{\myFrameBigFigure}[4][image]
{
\begin{figure}[ht!bp]
%\begin{figure}[H]
	\renewcommand{\figurename}{Obrázok}
	\checkoddpage
	\ifcpoddpage
		%nothing
	\else
		\hspace{-\marginparsep}\hspace{-\marginparwidth}
	\fi
	%use minipage to center the label beneath the figure
	\begin{minipage}{\fullwidth}
	\frame{%
		\includegraphics[width= \fullwidth]{#2}%
		}
		\caption[#4]{#3}
		\label{#1_#2}
	\end{minipage}
\end{figure}
}

%----------------------------------------------------------------------------------
% \myHUGEFigure	[ LABEL_PREFIX (optional) ]
%				{ FILENAME (without extension) }
%				{ CAPTION TEXT }
%				{ SHORT VERSION OF CAPTION TEXT }
%
%Bild wird rotiert und quer in kompletter Breite gesetzt
%landscape picture using the full width of the rotated page
\newcommand{\myHugeFigure}[4][image]
{
\begin{sidewaysfigure}[t!bp]
	\checkoddpage
	\ifcpoddpage
		%nothing
		\vspace{\marginparsep}\vspace{\marginparwidth}
	\else
		%nothing
		\vspace{-\marginparsep}\vspace{-\marginparwidth}
	\fi
		\includegraphics[width= \textheight]{#2}
		\caption[#4]{#3}
		\label{#1_#2}
	
\end{sidewaysfigure}
}

%----------------------------------------------------------------------------------
% \myFigure	[ LABEL_PREFIX (optional) ]
%			{ FILENAME (without extension) }
%			{ CAPTION TEXT }
%			{ SHORT VERSION OF CAPTION TEXT }
%
%Bild wird in der Breite der textspalte gesetzt
%picture using the width of the text column
\newcommand{\myFigure}[6][image]%
{%
%\scalebox{0.5}{
\begin{figure}[#6]%
	\renewcommand{\figurename}{Obrázok}
	\begin{center}%
		\includegraphics[scale=#5]{#2}%
		\caption[#4]{#3}
		\label{#1_#2}%
	\end{center}%
\end{figure}%
}%


%----------------------------------------------------------------------------------
% \myImgRef	[ LABEL_PREFIX (optional) ]
%			{ LABEL OF THE IMAGE }
%
%referenziert das angegebene Bild
%reference to an image
\newcommand{\myImgRef}[2][image]%
{%
	\ref{#1_#2}%
}%

%----------------------------------------------------------------------------------
% \myBigTable	{ YOUR TABULAR DEFINITION }
%			{ CAPTION TEXT }
%			{ TABLE_LABLE }
%
%Tabelle wird in kompletter Breite gesetzt
%table using the full width of the page
\newcommand{\myBigTable}[3]%
{%
\begin{table}[htdp]%
	\checkoddpage%
	\ifcpoddpage%
		%nothing
	\else%
		\hspace{-\marginparsep}\hspace{-\marginparwidth}%
	\fi%
	\begin{minipage}{\fullwidth}%
		\begin{center}%
			#1%
			\caption{#2}%
			\label{#3}%
		\end{center}%	
	\end{minipage}%
\end{table}%
}%

%----------------------------------------------------------------------------------
% \myTable	{ YOUR TABULAR DEFINITION }
%			{ CAPTION TEXT }
%			{ TABLE_LABLE }
%
%Tabelle wird in der Breite der Textspalte gesetzt
%table using the width of the text column
\newcommand{\myTable}[3]%
{%
\begin{table}[htdp]%
	\begin{center}%
		#1%
		\caption{#2}%
		\label{#3}%
	\end{center}%	
\end{table}%
}%

%----------------------------------------------------------------------------------
% \myTxtRef	{ LABLE }
%
%Referenz auf Kapitel oder Abschnitte - gibt nummer und namen aus, z.B.: 5.3---"Yaddahyaddah"
%references chapters or sections, outputs number and title, e.g., 5.3---"Yaddahyaddah"
\newcommand{\myTxtRef}[1]
{%
	\ref{#1} ``\nameref{#1}''%
}

%----------------------------------------------------------------------------------
% \myTxtRefPP	{ LABLE }
%
%Referenz auf Kapitel oder Abschnitte - gibt nummer, namen und seiten aus, z.B.: 5.3---"Yaddahyaddah" (p. 45)
%references chapters or sections, outputs number and title, e.g., 5.3---"Yaddahyaddah"
\newcommand{\myTxtRefPP}[1]
{%
	\ref{#1} ``\nameref{#1}'' (p.~\pageref{#1})%
}

%----------------------------------------------------------------------------------
% \myUnderscore
%
%Setzt einen "schšnen" Unterstrich fŸr URLs
%typesets a 'nice' underscore for URLs
\newcommand{\myUnderscore}{$\underline{\hspace{0.5em}}$}

%----------------------------------------------------------------------------------
%\myTilde
%
%Setzt eine "schšne" Tilde fŸr URLs
%typesets a 'nice' tilde for URLs
\newcommand{\myTilde}{$\sim$}

%----------------------------------------------------------------------------------
% \myURL	{ TYPESET VERSION OF ANCHOR }
%			{ PRISTINE URL }
%			{ TYPESET VERSION OF URL }
%
%Setzt eine URL
%die typographisch "schšne" version erscheint in einer Fu§note,
%im Text erscheint der Ankertext, verlinkt ist die "echte" URL
%typesets a URL
%the typographically correct version appears as a footnote,
%the anchor appears in the text, the link points to the pristine URL
\newcommand{\myURL}[3]%
{%
	\textcolor{blue}{%
		\href{#2}{#1}%
	}%
	\footnote{#3}%
}

%----------------------------------------------------------------------------------
% \mySimpleURL	{ TYPESET VERSION OF ANCHOR }
%				{ PRISTINE URL }
%
%Setzt eine URL
%die URL erscheint in einer Fu§note,
%im Text erscheint der Ankertext, die URL ist verlinkt
%typesets a URL
%the URL appears as a footnote,
%the anchor appears in the text, the link points to the URL
\newcommand{\mySimpleURL}[2]%
{%
	\textcolor{blue}{%
		\href{#2}{#1}%
	}%
	\footnote{#2}%
}

%----------------------------------------------------------------------------------
% \myProjectURL	{ TYPESET VERSION OF ANCHOR }
%				{ PRISTINE URL INSIDE PROJECT DIRECTORY }
%				{ TYPESET VERSION OF URL INSIDE PROJECT DIRECTORY }
%
%Setzt eine URL auf hci/public wo die Inhalte des WebServer Ordners auf "oliver" verfŸgbar sind
%die typographisch "schšne" version erscheint in einer Fu§note,
%im Text erscheint der Ankertext, verlinkt ist die "echte" URL
%typesets a URL to hci/public from where the contents of the WebServer folder from oliver can be accessed
%the typographically correct version appears as a footnote,
%the anchor appears in the text, the link points to the pristine URL
\newcommand{\myProjectURL}[3]%
{%
	\textcolor{blue}{%
		\href{http://hci.rwth-aachen.de/public/#2}{#1}%
	}%
	\footnote{http://hci.rwth-aachen.de/public/#3}%
}

%----------------------------------------------------------------------------------
% \mnote	{ MARGIN NOTE }
%
%Setzt eine Randnotitz
%puts a comment into the margin in small sans-serif font
\newcommand{\mnote}[1]%
{%
	\leavevmode%
	\checkoddpage%
	\ifcpoddpage%
		\marginpar{\raggedright\textsf{{\footnotesize{#1}}}}%
	\else%
		\marginpar{\raggedleft\textsf{{\footnotesize{#1}}}}%
	\fi%
}
	
% leavevmode allows mnotes to be aligned with the first line of a paragraph
% NOTE: you have to put a "%" at the end of the line with the mnote, or you will get an extra blank at the beginning of the paragraph!

%----------------------------------------------------------------------------------
% \todo	{ TODO MARGIN NOTE }
%
%Setzt eine "ToDo"-Randnotitz in rot zur Erinnerung
%puts a 'todo' comment into the margin in red
\definecolor{red}{rgb}{1,0,0}
\newcommand{\todo}[1]{\mnote{\textcolor{red}{ToDo: #1}}}

%----------------------------------------------------------------------------------
% \chapterquote	{ QUOTATION }
%				{ SOURCE }
%
%Setzt ein Zitat zum Einleiten eines Kapitels
%outputs a quote with its source, can be used as an introduction to chapters
\newcommand{\chapterquote}[2]{
\begin{quotation}
    \begin{flushright}
	\noindent\emph{``{#1}''\\[1.5ex]---{#2}}
    \end{flushright}
\end{quotation}
}

%----------------------------------------------------------------------------------
% \myDefBox	{ TERM }
%			{ DEFINITION }
%
%Setzt eine Randnotitz und eine farbige Box (Textspaltenbreite),
%welche einen Begriff und seine Definition enthŠlt
%outputs a margin note and a colored box (width of the text column) containing a term and its definition
\newcommand{\myDefBox}[2]
{%
	\setlength{\fboxrule}{1mm}%
	\fcolorbox{orange_med}{orange_light}%
	{%
		\parbox{\myDefBoxWidth}{{\bfseries\scshape#1:}\\#2}%
	}%
	\mnote{Definition:\\\emph{#1}}
}

%----------------------------------------------------------------------------------
% \myBigDefBox	{ TERM }
%				{ DEFINITION }
%
%Setzt eine farbige Box (Seitenbreite), welche einen Begriff und seine Definition enthŠlt
%outputs a colored box (width of the page) containing a term and its definition
\newcommand{\myBigDefBox}[2]
{%
	\begin{figure}[h!]
	\setlength{\fboxrule}{1mm}%
	\checkoddpage%
	\ifcpoddpage%
		%nothing
	\else%
		\hspace{-\marginparsep}\hspace{-\marginparwidth}%
	\fi%
	\fcolorbox{orange_med}{orange_light}%
	{%
		\parbox{\myBigDefBoxWidth}{{\bfseries\scshape#1:}\\#2}%
	}%
	\end{figure}
}

%----------------------------------------------------------------------------------
% \myDownloadURL	{ TYPESET DOWNLOAD NAME }
%					{ PRISTINE VERSION OF FILENAME }
%					{ TYPESET VERSION OF FILENAME }
%
%Setzt eine farbige Box, welche einen Downloadlink enthŠlt
%outputs a colored box containing a download link
\newcommand{\myDownloadURL}[3]{%
\checkoddpage%
	\ifcpoddpage%
		%nothing
	\else%
		\hspace{-\marginparsep}\hspace{-\marginparwidth}%
	\fi%
\setlength{\fboxrule}{1mm}%
\fcolorbox{green_med}{green_light}{%
\begin{minipage}{\myBigDefBoxWidth}%
\begin{center}%
\myProjectURL{#1}{folder/#2}{folder/#3}%
\end{center}%
\end{minipage}%
}%
}

%----------------------------------------------------------------------------------
% \emptydoublepage
%
% Leere Doppelseite ohne Kopf- oder Fu§zeile am Ende von Kapiteln
% Clear double page without any header or footer at end of chapters
\newcommand{\emptydoublepage}{\clearpage\thispagestyle{empty}\cleardoublepage}
\newcommand{\emptysinglepage}{\clearpage\thispagestyle{empty}\clearpage}

%----------------------------------------------------------------------------------
% \pagebreak	[ SOME STRANGE LATEX VALUE ]
%
%Eklige pagebreaks fŸr den Druck (falls es nicht mehr anders geht)
%pagebreaks for the final print version (last resort weapon against wrong pagebreaks by LaTeX)
\newcommand{\PB}[1][3]
{%
	\pagebreak[#1]%
}

%----------------------------------------------------------------------------------
% \TM
%
%Setzt ein (TM) Symbol
%Places a (TM) symbol
\newcommand{\TM}
{%
	\textsuperscript{\texttrademark}%
}


% zalamovanie slov
\hyphenation{
Twitter
Twittru
map-reduce
oblastiach
korpusov
processing
modelu
vstupno
ZeroMQ
tieto
priamo
vtedy
používateľom
udalostí
}


%--------------------------------------------------------------
\begin{document}

% allow more flexible whitespaces to avoid hyphenation and overfull hboxes
\sloppy %lepsi algoritmus na zalamovanie riadkov

% 'see'-entries for the index
\include{indexAddOn}


%--------------------------------------------------------------
% uvodne strany... obal, titulka, abstrakt, uvod, ...
%--------------------------------------------------------------
\frontmatter

%use the titlepage from the 'images' directory 
\newgeometry{rmargin=20mm}
\begin{titlepage}
% centrovanie
\begin{center}                                                                                                                                                   
{\Large Slovenská technická univerzita v Bratislave} \\
{\Large Fakulta informatiky a informačných technológií} \\
\vspace*{1\baselineskip}
% cislo prace
\large {}
\vfill % vert. plnenie prazdnymi medzerami                                                                                                                                                                    

%\newpage

{{Bc. Matúš Cimerman}} \\
\vspace*{1\baselineskip}
{\LARGE {\textbf{Analýza prúdu prichádzajúcich udalostí použitím rôznych metód pre analýzu údajov}}} % nadpis
%{\huge {Analýza prúdu údajov}} % nadpis
\\
\vspace*{1\baselineskip}
%{\Large {Bakalárska práca}}\\
\textit{Diplomová práca}\\
\vfill %vert. plnenie
\end{center}
% end centrovanie
{Vedúci práce: Ing. Jakub Ševcech}\\
\\
%\vspace*{1\baselineskip}
{máj, 2016}
\end{titlepage}
\emptydoublepage

\begin{titlepage}
% centrovanie
\begin{center}                                                                                                                                                   
{\Large Slovenská technická univerzita v Bratislave} \\
{\Large Fakulta informatiky a informačných technológií} \\
\vspace*{1\baselineskip}
% cislo prace
\large {}
\vfill % vert. plnenie prazdnymi medzerami                                                                                                                                                                    


{{Bc. Matúš Cimerman}} \\
\vspace*{1\baselineskip}
{\LARGE {\textbf{Analýza prúdu prichádzajúcich udalostí použitím rôznych metód pre analýzu údajov}}} % nadpis
%{\huge {Analýza prúdu údajov}} % nadpis
\\
\vspace*{1\baselineskip}
%{\Large {Bakalárska práca}}\\
\textit{Diplomová práca}\\
\vfill %vert. plnenie
\end{center}
% end centrovanie
{Študijný program: Informačné systémy}\\
{Študijný odbor: 9.2.6 Informačné systémy}\\
{Miesto vypracovania: Ústav informatiky, informačných systémov a softvérového inžinierstva, FIIT STU v Bratislave}\\
{Vedúci práce: Ing. Jakub Ševcech}\\
\\
{máj, 2016}

\end{titlepage}
\restoregeometry
\thispagestyle{empty}
\emptydoublepage

%\include{declarationofownwork}
%\emptydoublepage

%pred abstrakt treba zadanie
\includepdf[pages={1}]{navrh_1.pdf}
\includepdf[pages={1}]{navrh_2.pdf}

\includepdf[pages={1}]{zadanie.pdf}


%the abstract should include an english and a slovak version
%!TEX root = ./main.tex

% This file is part of the i10 thesis template developed and used by the
% Media Computing Group at RWTH Aachen University.
% The current version of this template can be obtained at
% <http://www.media.informatik.rwth-aachen.de/karrer.html>.

\loadgeometry{myAbstract}

\chapter*{Abstract\markboth{Abstract}{Abstract}}
\addcontentsline{toc}{chapter}{\protect\numberline{}Abstract}
\label{abstract}

%here (english version)
Nowadays we can see emerging need for data analysis as data occur. Processing and analysis of data streams is a complex task, first, we particuraly need to provide low latency and fault-tolerant solution.

In our work we focus on proposal a set of tools which will help domain expert in process of data analysis. Domain expert do not need to have detailed knowledge of analytics models. Similar approach is popular when we want analyse static collections, eg. funnel analysis. We study possibilities of usage well known methods for static data analysis in domain data streams analysis. Our goal is to apply method for data analysis in domain of data streams. This approach is focused on simplicity in use of selected method and interpretability of results. It is essential for domain experts to meet these requirements because they will not need to have detailed knowledge from such a domains as machine learning or statistics. We evaluate our solution using software component implementing chosen method.
\\
\emptydoublepage
\chapter*{Abstrakt\markboth{Abstrakt}{Abstrakt}}
\addcontentsline{toc}{chapter}{\protect\numberline{}Abstrakt}
\label{abstrakt}
%here (slovak version)
Dnes môžeme pozorvať narastujúcu potrebu analyzovať dáta počas ich vzniku. Spracovanie a analýza prúdov dát predstavuje komplexnú úlohu, pričom je dôležité poskytnúť riešenie s nízkou odozvou, ktoré je odolné voči chybám.

V našej práci sa sústreďujeme na návrh súboru nástrojov, ktoré pomôžu doménovému expertovi počas analýzy dát.
%Slovenska verzia.
\emptydoublepage
\loadgeometry{myText}

\emptydoublepage

%acknowledgements
\loadgeometry{myAbstract}

\chapter*{Poďakovanie\markboth{Poďakovanie}{Poďakovanie}}
%\addcontentsline{toc}{chapter}{\protect\numberline{}Poďakovanie}

\vfill
Na prvom mieste vyslovujem poďakovanie vedúcemu mojej diplomovej práce, Ing. Jakubovi Ševcechovy, za všetky jeho odborné rady, odovzdané skúsenosti a usmernenie pri tvorení práce. \\
Touto cestou taktiež vyslovujem poďakovanie všetkým výskumníkom zo skupiny PeWe, za prínosné diskusie a ich spätnú väzbu týkajúcu sa mojej práce. \\
V neposlednom rade ďakujem celej mojej rodine a priateľom.

\begin{flushright}
Matúš Cimerman
\end{flushright}


\loadgeometry{myText}


\emptydoublepage

%conventions applied in the thesis
%\include{conventions}
%\emptydoublepage

\renewcommand{\contentsname}{Obsah}
\singlespacing
\tableofcontents

\emptydoublepage
\onehalfspacing
%\doublespacing
%
%\listoffigures
%\emptydoublepage
%
%\listoftables
%\emptydoublepage

\newtheorem{definition}{Definícia}[section]
\newtheorem{hypothesis}{Hypotéza}[section]


%--------------------------------------------------------------
% hlavny obsah
%--------------------------------------------------------------
\mainmatter

%!TEX root = ./main.tex

\chapter{Úvod}
%\label{introduction}
V súčasnosti pozorujeme zvýšený záujem o oblasť analýzy a dolovania dát. Vhodné použitie a výber metód pre spracovanie dát prináša hodnotné výstupy a náhľady pre používateľa. Výstupy môžu byť použité pre strategické rozhodnutia v podnikoch. Najčastejší postup je aplikovaním metód ako napríklad lieviková analýza alebo rozhodovacie stromy nad statickou kolekciu dát. Tento prístup má niekoľko problémov a to najmä: všetky trénovacie dáta musia byť uložené v pamäti alebo na disku, spracovanie a výpočtová náročnosť a vysporiadanie sa s trendami a zmenami v dátach. Nutnosť dáta najskôr zozbierať a uložiť, čo je dnes, kedy vznikajú milióny záznamov za deň, či hodinu, predstavuje rovnako veľký problém.\\
\\
Pod pojmom spracovanie v reálnom čase myslíme spracovanie v takmer reálnom čase, tzv. jemné (angl. soft) spracovanie v reálnom čase. Jemné spracovanie v reálnom čase znamená, že systém negarantuje spracovanie a odpoveď v stanovenom časovom limite, pričom niektoré vzorky sa môžu omeškať alebo úplne vynechať \citep{stankovic1988real}. Presné limity, do kedy sa spracovanie považuje za reálny čas závisí od problému. Niekde to môže predstavovat rádovo stotiny sekundy, v inej úlohe rádovo sekundy. V tejto práci budeme pracovať s pojmom spracovanie v reálnom čase chápajúch ako jemné spracovanie v reálnom čase.\\
\\
Pri dolovaní v prúde dát čelíme niekoľkým výzvam: objem,  rýchlosť (frekvencia) a rozmanitosť. Veľký objem dát, ktoré vznikajú veľmi rýchlo je potrebné spracovať v ohraničenom časovom intervale, často v reálnom čase. Pričom sa objem dát neustále zväčšuje, potenciálne narastá až do nekonečna. Identifikujeme niekoľko najviac zasiahnutých oblastí, ktoré sú zdrojmi týchto dát: počítačové siete, sociálne siete, Webové stránky (sledovanie správania používateľa na stránke) a Internet Vecí (angl. Internet of Things). Na informácie generované z takýchto zdrojov sa často pozeráme ako na neohraničené a potenciálne nekonečné prúdy údajov.\\
\\
Spracovanie, analýza a dolovanie v týchto prúdoch je komplexná úloha. Pre aplikácie je kritické spracovať údaje s nízkou odozvou, pričom riešenie musí byť presné, škálovateľné a odolné voči chybám. Nakoľko sú prúdy neohraničené vo veľkosti a potenciálne nekonečné, môžeme spracovať len ohraničený interval prúdu. Potom hovoríme, že dáta musia byť spracované tak ako vznikajú. Tradičné metódy a princípy pre spracovanie statickej kolekcii údajov nie sú postačujúce na takéto úlohy \citep{krempl2014open, han2011data}.
\emptydoublepage
%\chapter{Otvorené problémy pri analýze prúdu údajov}
\label{Otvorené problémy pri analýze prúdu údajov}
Dolovanie a objavovanie znalostí, alebo tiež analýza a spracovanie prúdu dát čelí trom hlavným výzvam: \textit{objem}, \textit{rýchlosť} a \textit{nestálosť} dát \citep{krempl2014open}. 

\section{Použitie tradičných metód dolovania dát a strojového učenia}

\section{Evaluácia modelov a algoritmov}

\section{Vizualizácia výsledkov používateľovi}

\section{Technické výzvy}

%\emptydoublepage
\include{methods}
\emptydoublepage
\chapter{Analytické úlohy nad prúdom dát}
\label{Analytické úlohy v prúde dát}
Spracovanie, analýza a dolovanie dát predstavuje vo všeobecnosti výzvu. Zvláštnu pozornosť si tieto úlohy vyžadujú pri spracovaní, analýze a dolovaní z prúdu prúdu udalostí. Prúd udalostí je často nazývaný \textit{prúd dát} alebo \textit{údajov}, či len skrátene \textit{prúd}. V tomto texte budeme pre jednoduchosť používať najmä termín \textit{prúd} a \textit{prúd dát} \citep{tran2014change}. Avšak môžu sa vyskytnúť aj terminý ako \textit{prúd udalostí}, \textit{sekvencia udalostí, či elementov}, pričom všetky termíny majú v tomto texte \textit{rovnaký} význam. \par
\begin{definition}{Prúd je potenciálne nekonečná sekvencia elementov \citep{tran2014change}.}
\begin{align*}
	S = \{(X_1,T_1), ..., (X_j,T_j), ...\}
\end{align*}
Kde každý element je pár $(X_j,T_j)$ kde $X_j$ je d-dimenzionálny vektor $X_j = (x_1, x_2, ..., x_d)$ prichádzajúci v čase $T_j$. $T_j$ je často nazývaný aj časová pečiatka, existujú dva typy časovej pečiatky: explicitná je generovaná keď dáta dorazia, implicitná je priradená vektoru v čase ich vzniku.
\end{definition}

% Vseobecny popis toho preco je potrebne prudove spracovanie dat
Takmer každé odvetvie dnes generuje masívne množstvo dát. Vzhľadom na ich veľký objem analytyci a doménový experti často strácajú schopnosť dolovať v celej sade dát. Stáva sa preto častým zvykom, že sa vyberie reprezentujúca vzorka, ktorej spracovanie predstavuje menšiu časovú a pamäťovú výzvu. Pri pamäťovej náročnosti hovoríme o limitoch počítača, pričom ak hovoríme o časovej náročnosti hovoríme o limitovanom čase doménového experta (čakanie na výsledok analýzy) \citep{hulten2001mining}. Predpokladajme, že bude pre doménoveho experta vysokým prínosom možnosť vykonávať analýzy nad prúdom v reálnom čase. Výstupy z takejto analýzy sú na rôznej granularite a úrovni, pričom môžu byť neskôr použité na ďalšie spracovanie alebo na priame prezentovanie výsledkov. \par

% Vyskumne vyzvy v prudoch dat vseobecne
Analýza a spracovanie prúdov dát pridáva viaceré otvorené výzvy a možnosti pre výskum \citep{krempl2014open}:
\begin{itemize}
	\item \textit{Ochrana súkromia a dôvernosti} pri analýze a dolovaní v prúde dát. Hlavným cieľom je vyvinúť metódy a techniky, ktoré neodhalia informácie a vzory, ktoré by kompromitovali potreby dôvernosti a ochrany súkromia. Dve hlavné výzvy pri analýze a dolovaní v prúdoch dát sú: \textit{vysporiadanie sa s neúplnými dátami} a \textit{uchovanie zmien (angl. concept drift) v prúde dát}.
	\item \textit{Predspracovanie} dát je dôležitou súčasťou každej reálnej aplikácie, najmä tých pre analýzu dát. Zatiaľ čo pri tradičnej analýze dát je predspracovanie vykonané jednorázovo, zvyčajne doménovým expertom. ktorý rozumie dátam. Pri prúde dát toto nieje prijateľné, pretože dáta nepretržite prichádzajú. Okrem niekoľkých štúdií \citep{zliobaite2014adaptive, anagnostopoulos2008deciding} tejto problematike nebola venovaná dostatočná pozornosť ako pri tradičnom spracovaní dát. Hlavné výzvy, ktorým treba čeliť pri predspracovaní prúdu dát sú: \textit{hluk v dátach}, \textit{outliers} a \textit{adaptívny výber vzorky}.
	\item \textit{Načasovanie a dostupnosť informácie}, väčšina algoritmov robí jednoduchý predpoklad, že prijatá informácia je kompletná, ihneď dostupná, prijatá pasívne a zadarmo. Viaceré výzvy spojené s načasovaním a dostupnosťou informácie sú formulované a nepreskúmané: \textit{spracovanie nekompletných dát}, \textit{vysporiadanie sa so skreslenou (angl. skewed) distribúciou dát} a \textit{spracovanie oneskorených dát}.
	\item \textit{Dolovanie entít a udalostí} kde entity predstavujúce prúd sú spojené do viacerých inštancií resp. štruktúrovaných informácií (napr. agregácie). Tieto entity môžu byť niekedy spojené s výskytom udalostí resp. v prúde dát. 
	\item \textit{Evaluácia algoritmov pre prúdy dát} predstavuje úplne novú výzvu v porovnaní s tradičnými metódami. Pri evaluácií v prúde dát sa musíme vysporiadať s problémami ako: \textit{zmeny (angl. concept drift}, \textit{limitovaný čas pre spracovanie vzorky}, \textit{vyvíjajúce sa skreslenie tried dát}, či \textit{oneskorenie overenia}. Tejto problematike sa v poslednej dobe venuje vyššia pozornosť, ako napríklad pre evaluáciu klasifikátorov nad prúdmi dát \citep{bifet2015efficient}.
	\item \textit{Špecializované, reaktívne a jednoduché modely} na pochopenie pre doménového experta. Tieto tri výzvy v sebe ukrývajú potrebu pre minimalizáciu závislosti na nastavení parametrov metódy, kombinácia online a offline modelov a riešenie správného problému (zmeny v prúdoch).
\end{itemize}

% Model prudu dat
\paragraph{Model prúdu dát} môže byť jeden z nasledujúcich: model časových radov, pokladničný model a model turniketu. Podľa modelu prúdu dáť existujú príslušné algoritmy, ktoré boli vytvorené pre daný model \citep{tran2014change}. Majme prúd dát $a_1, a_2, ...$, ktorý prichádza sekvenčne za sebou a popisuje podstaný signál $A$. 
V modeli časových radov každá vzorka $a_i$ sa rovná $A[i]$ pričom vzorky prichádzajú v vzostupnom poradí. Tento model je vhodný pre prúdy dát, ktoré nesú v sebe časovú postupnosť alebo je ich poradie určované časovou pečiatkou \citep{muthukrishnan2005data}.
Pri pokladničnom modeli môžme považovať množinu $U = {1, 2, ..., n}$ za element z prúdu dát. Ak uvažujeme sekvenciu $2, 1, 2, 5$ ako príklad, potom hovoríme o pokladničnom modeli. Tento model je často používaný v praxi, napríklad v prípadoch kde sled IP adries pristupuje na Web server \citep{ikonomovska2013algorithmic, muthukrishnan2005data}.
Model turniketu je veľmi podobný pokladničnému modelu. Rozdiel je v tom, že vzorka môže predstavovať aj zápornú hodnotu - analógia z reálneho sveta kedy niektorí ľudia prichádzajú a vychádzajú turniketom, počet ľudi sa mení (napr. na zjazdovke) \citep{ikonomovska2013algorithmic, muthukrishnan2005data}.

% Predspracovanie prudu dat
\paragraph{Predspracovanie prúdu}
Predspracovanie je azda najdôležitejším krokom v aplikáciach reálneho sveta a časovo najnáročnejšou úlohou pre každého analytika. Nakoľko dáta prichádzajú z nehomogénneho sveta, môžu byť zašumené, nekompletné, duplicitné alebo často obsahovať hodnoty, ktoré sa značne líšia od ostatných. Predspracovanie prúdiach údajov je potrebné čo najviac automatizovať. Existuje niekoľko známych metód a techník, ktoré sú používané pri predspracovaní prúdov dát:
\begin{itemize}
	\item \textit{Vzorkovanie}, napríklad podľa pravdepodobnostného modelu.
	\item \textit{Zahadzovania potenciálne nepotrebných vzoriek}, ak je spracujúci proces príliš zaťažený. Tu môže nastať problém, že práve zahodená vzorka bola dôležitá (zmena v dátach).
	\item \textit{Agregácia} údajov môže značne znížiť objem dát, ale môže spôsobiť problém pri potrebe pohľadu do minulosti.
	\item \textit{Aproximačné algoritmy} a ich použitie má za následok podstatné zrýchlenie spracovania a analýzy prúdov za predpokladu istej chybovosti. Chybovosť je zvačsa ohraničená.
	\item \textit{Posuvné okno}, tento prístup vznikol s potrebou analýzy definovaného časového okna z prúdiacih údajov.
\end{itemize}

% Dolovanie a extrakcia informacii
\paragraph{Spracovanie, dolovanie a analýza informácií} 
Problematika spracovania, dolovania a analýzy bola študovaná niekoľko dekád. Zvýšenú pozornosť začala odborná verejnosť venovať pri aplikovaní týchto úloh na prúdy dát. Týmto úlohám sa venujeme podrobne v nasledujúcich podkapitolách:
\begin{itemize}
	\item \textit{Klastrovanie}, existuje niekoľko výskumov, ktoré sa venovali špeciálne klastrovaniu implementovaním napríklad k-mediánu a inkrementálnych algoritmov.
	\item \textit{Klasifikácia}, táto úloha je dlho skúmaná s použitím rôznych metód rozhodovacích stromov.
	\item \textit{Počítanie frekvencie a opakovaní}, použitím posuvných okien a inkrementálnych algoritmov na detekciu vzorov v prúde.
	\item \textit{Analýza časových radov použitím symbolickej} reprezentácie časových radov v prúde dát. Takáto reprezentácia nám umožňuje redukciu veľkosti prenášaných dát. Táto technika pozostáva z dvoch hlavných krokov, aproximácia po častiach a následná transformácia výsledku do diskrétnych veličín.
\end{itemize}


%%%%%%%%%%%%%%%%%%%%%%%%%%%%%%%%%%%%%%%%%%%%%%%%%
%         Dopytovanie sa v prúdoch dát          %
%%%%%%%%%%%%%%%%%%%%%%%%%%%%%%%%%%%%%%%%%%%%%%%%%
\section{Dopyty nad prúdom dát}
Vyhodnocovaniu dopytov nad statickou kolekciou dát bola venovaná značná pozornosť, ak však hovoríme o prúdoch dát dopyty musia byt vyhodnocované kontinuálne \citep{babu2001continuous, babcock2002models}. Vzniká teda nová paradigma pre interakciu s dynamicky sa meniacimi dátami, ktorú nazývame kontinuálne dopyty (angl. continious queries) \citep{babu2001continuous}. Výsledky kontinuálnych dopytov sú produkované dynamicky v čase vzniku nových dát. Príkladom použitia takýchto dopytov je napríklad sledovanie vývoja akcií burzy. Problém môže nastať pri jednorázových dopytoch, ktoré obsahujú agregačné funkcie. Pri tradičnom spracovaní dát kde sú všetky dáta uložené ako statická kolekcia, je dopyt vykonaný nad celou kolekciou. V prípade kontinuálneho dopytu je problém získať predchádzajúce dáta za predpokladu, že dáta niesú ukladané. Môžu potom nastať dva scenáre:
\begin{enumerate}
	\item agregačná funkcia je prepočítaná nad kolekciou dát, za predpokladu, že boli historické dáta ukladané.
	\item agregačná funkcia je počítaná od momentu zadanie dopytu.
\end{enumerate}
Kontinuálne dopytovanie do prúdu dát nesie so sebou nieľko výziev \citep{babcock2002models}:
\begin{itemize}
	\item \textit{Limitované pamäťové požiadavky} na algoritmy spracujúce dopyty, pretože prúd dát predstavuje potenciálne nekonečný prúd udalostí.
	\item \textit{Približné odpovede na dopyty} sú niekedy postačujúce za predpokladu, že odpoveď je dostatočné rýchla a používateľ rozumie v akej presnosti mu bola odpoveď poskytnutá. Techniky pre redukciu dimenzionality a objemu dát zahŕňajú napríklad: histogramy, náhodné vzorkovanie, symbolické vzorkovanie apod.
	\item \textit{Dopytovací jazyk} by mal byť podobný štandardu jazyka SQL. Jazyk SQL je známy deklaratívny jazyk, je široko používaný so zavedením štandardom, ktorý poskytuje flexibilitu a optimálnu evaluáciu dopytu a vykonanie nad prúdom, či datasetom. 
\end{itemize}
Výskumné práce sa tiež venovali adaptívnym kontinuálnym dopytom nad prúdmi dát. Bolo ukázané, že takýto prístup môže mať značný prínos v oblasti výkonnosti systému vďaka jeho schopnosti adaptácie na zmeny v prúde dát. Tieto vlastnosti sú dosiahnuté aplikovaním zoskupovania indexov filtrov na priebežný výber predikátov \citep{madden2002continuously}. \par
Ďalší priestor na zlepšenie výkonnosti kontinuálnych dopytov nad prúdmi dát predstavujú adaptívne filtre. Pri dopytovaní sa takmer vždy vykonáva filtrovanie dát v nejakej podobe. Tento krok filtrovania je obvykle implementovaný v systéme na spracovanie dopytov. Pre zvýšenie výkonnosti dopytov je preto možné tieto filtre presunúť priamo do zdrojov dát. Ukázalo sa, že takýto prístup môže mať pozitívny dopad na výkonnosť \citep{olston2003adaptive}. Tento prístup prinesie najmä redukciu prenášaných dát výmenou za ich nepresnosť. Problémom tejto techniky je, že je aplikovateľná len v prostredí, ktoré máme plne pod kontrolou a vieme zasahovať do všetkých jeho súčastí.



%%%%%%%%%%%%%%%%%%%%%%%%%%%%%%%%%%%%%%%%%%%%%%%%%
%               Detekcia zmien                  %
%%%%%%%%%%%%%%%%%%%%%%%%%%%%%%%%%%%%%%%%%%%%%%%%%
\section{Detekcia zmien}
Detekcia zmien (angl. concept drift) zohráva, v dnešnom rýchlo sa meniacom svete, dôležitú úlohu. Zmeny nastávajú veľmi rýchlo a nečakane. Preto stúpa potreba detekcie zmeny a následná správna reakcia, ktorá vyplynie z detekovanej zmeny. Na to aby sme boli schopný na tieto zmeny adekvátne reagovať je potrebné dáta spracovávať tak ako vznikajú a pozerať sa na ne ako na prúd udalostí. Tradičné metódy pre paralelné spracovanie uvažujú len statickú kolekciu dát \citep{tran2014change}. \par
% TODO: pridat graf, ktory znazornuje rozne zmeny (nahla, inkrementalna, ...)
Detekcia zmeny predstavuje proces identifikácie zmeny aktuálneho stavu modelu voči predchádzajúcemu. Na tento objekt sa pozeráme v rôznom čase. Dôležitý rozdiel medzi zmenou a rozdielom je, že zmena hovorí o prechode modelu do iného stavu, zatiaľ čo rozdiel znamená nepodobnosť v atribútoch dvoch objektov. V kontexte prúdu, detekovanie zmeny je proces segmentácie prúdu udalostí do rôznych segmentov a identifikovanie miest kde sa zmení dynamika prúdu \citep{ross2009online}. Metóda pre detekciu zmien musí riešiť nasledujúce úlohy \citep{tran2014change}: \textit{detekcia zmeny} znamená správnu identifikáciu zmeny a \textit{lokalizácia zmeny} hovorí o identifikovaní momentu kedy zmena nastala. Týmto úloh je potrebné venovať dostatočnú pozornosť, pretože zmeny môžu byť falošné alebo dočasné čo so sebou prináša problém lokalizácie danej zmeny. Ďalší rozdiel, ktorý je potrebné zadefinovať, je medzi rozdiel detekovaním posunu pojmu (angl. concept drift). Pre lepšiu čitateľnosť tohto textu budeme pod pojmom detekcia zmeny, zmena rozumieť posun pojmu. Detekcia concept drift-u sa sústreďuje na označkované dáta, zatiaľ čo detekcia zmeny pracuje s označkovanými rovnako ako s neoznačkovanými dátami. Posun pojmu nazývame tiež časté zmeny v účelovej funkcii modelu, ktorý sa učí online. \par

Metódy pre detekovanie zmien môžme klasifikovať do nasledujúcich prístupov \citep{liu2010mining}: \textit{metódy založené na stave}, \textit{metódy sledujúce trend} a \textit{prahové metódy}. Algoritmus pre detekciu zmien by mal spĺňať aspoň nasledovné požiadavky: \textit{presnosť}, \textit{rýchlosť} a \textit{odpoveď v reálnom čase}. Algoritmus by tiež mal detekovať čo najmenej chybných zmien a čo najviac správnych presných miest zmeny. Algoritmy by mali byť prispôsobené reálnemu prostrediu a spracovaniu prúdov vysokých objemov a rýchlostí. Na obrázku \ref{fig:zmeny-vseobecny-dia} je zobrazený všeobecný diagram pre detekciu zmeny v prúde udalostí.

\myFigure{images/2_zmeny_vseobecny-diagram}{Všeobecný diagram zobrazujúci detekciu zmeny v prúde udalostí \citep{tran2014change}.}{test-dia}{0.5}{h!}\label{fig:zmeny-vseobecny-dia}

% Techniky a metody pre detekciu zmeny v prudoch
Pre detekciu zmeny v prúdoch dát bolo vyvinutých niekoľko techník a metód. Niektoré z nich nižšie podrobnejšie popisujeme.

% Charakteristika dat
\paragraph{Charakteristika dát} Metódy pre detekciu zmien môžu byť klasifikované na základe charakteru dát, s ktorými pracujú. Najčastejšie môžme prúdy klasifikovať do kategorických alebo numerických prúdov. Ak hovoríme o kategorických prúdoch, dáta obsiahnuté v prúde majú kategorický charakter, napríklad rôzny výrobcovia áut: $x \in \{Volvo, Toyota\}$. Pri numerických prúdoch dáta predstavujú numerické hodnoty $x \in {\rm I\!R}$. Pre každý takýto prúd boli vyvinuté príslušné algoritmy. Problém nastáva pri aplikáciach s dátami reálneho sveta kde prúdy často obsahujú numerické aj kategorické dáta. V takýchto situáciach má zmysel dáta rozdeliť rovnomenných skupín obsahujúce dáta rovnakého typu. Na tieto skupiny sú následne použté príslušné algoritmy. Prúdy dát sa ďalej môžu klasifikovať do označkovaných a neoznačkovaných prúdov. Neoznačkované prúdy obsahujú dáta, ktoré niesú zaradené do žiadnej triedy. Naopak označkované prúdy nesú v sebe informáciu o tom, do ktorej triedy patrí vybraný element. Rôzny charakter prúdu predstavuje rôzne zmeny a prístup na ich riešenie pri detekcii zmien v prúde \citep{tran2014change}.

% Kompletnost statistickej informacie
\paragraph{Metóda pre detekciu zmeny} V skratke DDM z anglického Drift Detection Method. Táto metóda sa zaoberá detekciou zmeny modelu. Majme prúd dát $(x_i,y_i)$ kde $x_i$ predstavuje atribúty a $y_i$ triedu vzorky. Model sa potom snaží predikovať skutočnú triedu $y_i+1$ novej vzorky. Gama a spol. založili DDM na fakte, že každá iterácia klasifikátora predikuje triedu vzorky. Klasifikátor je binárny, takže trieda môže byť len $pravda$ alebo $nepravda$. Potom, pre množinu vzoriek, chyba predstavuje náhodnú premennú z Bernoulliho pokusov (angl. Bernoulli trials). Vďaka tomu môžme chybu modelovať s bínomickým rozdelením. Nech $p$ je pravdepodobnosť zlej predikcie a $s_i$ je štandardná odchýlka vypočítaná nasledovne:
\begin{align*}
s_i = \sqrt{ \frac{p_i(1-p_i)} {i} }
\end{align*}
Pre každú vzorku z prúdu sú udržiavané dve premenné, $p_min$ a $s_min$. Ich hodnoty sú použité na výpočet varovnej hodnoty, ktorá slúži na definovanie optimálnej velkosti kontextového okna. Kontextové okno si udržiava staré vzorky, ktoré obsahujú nový kontext resp. zmenu, či posun pojmu, a minimálny počet elementov zo starého konextu. Ak sa následne zníži množstvo chybne predikovaných vzoriek, okno je zahodené ako zle identifikovaná zmena (false alarm). Naopak, ak je dosiahnutá dostatočná varovná úroveň, predtým naučený model je zahodený a vytvorený nový, ale iba zo vzoriek ktoré boli uložené do kontextového okna \citep{gama2004learning, brzezinski2010mining}.
\par
Existuje tiež rozšírenie EDDM, ktoré je modifikáciou DDM. Tento algoritmus používa rovnakú techniku varovných alarmov, ale namiesto klasifikácie chyby používa metriku množstva rozdielnych chýb. EDDM metóda dosahuje lepšie výsledky pri postupných zmenách, ale je citlivejšia na hluk v dátach \citep{wadewale2015survey}.

% ADWIN
\paragraph{ADWIN} je skratka pre algoritmus s názvom adaptívne posuvné okno (angl. adapting sliding window). Tento algoritmus je vhodný je prúdy s náhlymi zmenami. Algoritmus si udržiava okno $W$ s najnovšími vzorkami. Okno $W$ je automatický zväčšované, ak nieje detekovaná žiadna výrazná zmena v prúde a naopak zmenšované, ak bola zmena detekovaná. Obmedzenie nárastu okna do nekonečna (žiadna zmena v prúde) je možné parametrom algoritmu, ktorý bude limitovať dĺžku okna $W$. ADWIN taktiež poskytuje ohraničenie výkonu na základe množstva falošne pozitívne a falošne negatívnych vzoriek \citep{wadewale2015survey}. Základná verzia algoritmu ADWIN je vhodná pre 1-dimenzionálne dáta. Ak je potrebné detekovať zmeny pre viac-dimenzionálne dáta, potom sa vytvára paralelne niekoľko okien pre každú dimenziu dát \citep{brzezinski2010mining}.

\par
Existuje mnoho ďalších prístupov ako sa vysporiadať so zmenami v prúde, napríklad: exponenciálne váhovaný posuvný priemer, štatistické testovanie rovnomerného podielu, súborové (angl. ensemble) metódy. Popis všetkých metód je nad rámec tejto práce.



%%%%%%%%%%%%%%%%%%%%%%%%%%%%%%%%%%%%%%%%%%%%%%%%%
%             Detekcia anomálií                 %
%%%%%%%%%%%%%%%%%%%%%%%%%%%%%%%%%%%%%%%%%%%%%%%%%
\section{Detekcia anomálií}
Detekcia anomálií (angl. anomaly detection) predstavuje proces identifikácie dát, ktoré sa význačne odchyľujú (angl. deviate) od historických vzorov \citep{hodge2004survey}. Anomálie môžu spôsobovať chyby v meraní senzorov, nezvyčajné správanie systému alebo chyba pri prenose dát, či zámerné vytváranie anomálií v používateľmi generovanom obsahu. 
Takže detekcia anomálií má veľa praktického použitia napríklad v aplikáciach, ktoré dohliadajú na kvalitu a kontrolu dát \citep{hill2007real} alebo adaptívne monitorovanie sietí  \citep{hill2010anomaly}. Tieto aplikácie často kladú požiadavku aby boli anomálie detekované v čase ich v vzniku, teda v reálnom čase. Potom metódy pre detekciu anomálií musia byť rýchle vo vykonávaní a mať inkrementálny charakter. \par

V minulosti sa obvykle anomálie detekovali manuálne s pomocou vizualizačných nástrojov, ktoré doménovým expertom pomáhali v tejto úlohe. Manuálne metódy avšak zlyhávajú pri detekcií anomálií v reálnom čase. Výskumníci navrhli niekoľko metód, ktoré majú myšlienku v prístupoch strojového učenia sa a automatizovaného štatistického vyhodnocovania \citep{hill2010anomaly}: \textit{minimálny objem elipsoidu}, \textit{konvexný zvon}, \textit{najbližší sused}, \textit{zhlukovanie}, \textit{klasifikácia neurónovou sieťou}, \textit{klasifikácia metódou podporných vektorov} a \textit{rozhodovacie stromy}. Tieto metódy sú pochopiteľne rýchlejšie než manuálna detekcia, avšak jeden význačný nedostatok, bez úpravy niesú vhodné pre prúdové spracovanie v reálnom čase. Existujú napríklad rozhodovacie stromy, ktoré si dokážu budovať model inkrementálne, avšak sa líšia od dobre známych algoritmov. Táto metóda je podrobne popísana ďalej v texte.

\paragraph{Dátovo riadená metóda} (angl. data-driven), ktorú navrhli \citep{hill2010anomaly}, využíva dátovo riadený jednorozmerný autoregresívny model prúdu dát a predikčný interval (ďalej len PI) vypočítaný z posledných historických dát na identifikáciu anomálií v prúde. Dátovo riadený model časového radu je použitý, pretože je jednoduchší na implementáciu a použitie v porovnaní s ostatnými modelmi časových radov. Tento model tiež poskytuje rýchle a presné prognózy. Dáta sú potom klasifikované ako anomálie na základe toho, či sú spadnú do zvoleného intervalu PI. Metóda teda poskytuje principiálny rámec pre výber hraničného prahu kedy majú byť anomálie klasifikované. Výhoda metódy je, že nevyžaduje žiadne vzorky dát, ktoré sú vopred označkované alebo klasifikované. Je veľmi dobre škálovateľná na veľké objemy dát a vykonáva inkrementálne počítanie tak ako dáta vznikajú.
Metóda pozostáva z nasledujúcich krokov so začiatkom v čase \textit{t}:
\begin{enumerate}
	\item použi model na predikciu o krok vpred (angl. one-step-ahead), ktorý má ako vstup $\displaystyle D^t = \{x_{t-q+1}, ..., x_t\}$ \textit{q} je rôzne meranie \textit{x} v čase \textit{t} a $\displaystyle D^t$ je model predikcie. Tento model je použitý ne predikovanie hodnoty $\displaystyle \overline{x}_{t+1}$ ako očakávaná hodnota v čase \textit{t+1}.
	\item výpočet hornej a spodnej hranice kam by malo spadnúť pozorované meranie s pravdepodobnosťou \textit{p}.
	\item porovnaj pozorovanie v čase \textit{t+1}, či spadá do určeného intervalu. Ak spadne mimo intervalu, objekt je klasifikovaný ako anomália.
	\item 
		\begin{enumerate}
			\item pri stratégii metódy detekcie anomálií a zmiernenia (angl. anomaly detection and mitigation) ADAM, ak je pozorovaný objekt klasifikovaný ako anomália, modifikuj $\displaystyle D^t$ odstránením $\displaystyle x_{t-q+1}$ z konca pozorovaného okna a pridaním $\displaystyle \overline{x}_{t+1}$ na začiatok okna, čím vytvoríme $\displaystyle D^{t+1}$.
			\item pri jednoduchej stratégii detekcie anomálií (angl. anomaly detection) AD, modifikuj $\displaystyle D^t$  odstránením $\displaystyle x_{t-q+1}$ z konca okna a pridaj $\displaystyle x_{t+1}$ na začiatok okna čím vznikne $\displaystyle D^{t+1}$.
		\end{enumerate}
	\item opakuj kroky \textit{1-4}
\end{enumerate}
\paragraph{Metóda dynamických bayesových sietí} (angl. Dynamic Bayesian Networks) \citep{hill2007real} bola vytvorená pre detekciu anomálií v prúdoch zo senzorov, ktoré sú umiestnené v životnom prostredí. Bayesové siete predstavujú acyklický orientovaný graf, zobrazené na obrázku \ref{fig:anomalie-dbn}, v ktorom každý uzol obsahuje pravdepodobnostú informáciu v súvislosti k všetkým možným stavom, v ktorých sa môže premenná nachádzať. Táto informácia spolu s topológiou bayesovej siete, špecifikuje úplné spojenie distribúcie stavu premennej, pričom sada známych premmených môže byť použitá na odvodenie hodnoty neznámych premenných. Dynamické bayesové siete s topológiou, ktorá sa vyvýja v čase, pridáva nové stavové premenné pre lepšiu reprezentáciu stavu systému v aktuálnom čase \textit{t}. Stavové premmné môžeme kategorizovať ako \textit{neznáme}, ktoré predstavujú skutočný stav systému a \textit{merané}, ktoré sú nedokonalé merania. Tieto premenné môžu byť naviac diskrétne alebo spojité. Nakoľko sa veľkosť siete zväčšuje s časom, vytváranie záverov použitím celej siete by bolo neefektívne a časovo náročné. Preto boli vyvinuté aproximačné algoritmy ako \textit{Kalmanové filtrovanie} alebo \textit{Rao-Blackwellized časticové filtrovanie}. \par
Hill et al. navrhli v \citep{hill2007real} dve stratégie pre detekovanie anomálií v prúde dát:
\begin{itemize}
	\item \textit{Bayesov dôveryhodný interval} (angl. Bayesian credible interval - BCI), ktorý sleduje viacrozmernú gausovskú distribúciu lineárneho stavu premennej, ktorý korešponduje s neznámym stavom systému a jej meraným náprotivkom.
	\item \textit{Maximálne posteriori meraný status} (angl. Maximum a posteriori measurement status - MAP-ms) používa komplexnejšiu dynamickú bayseovú sieť. Princíp je rovnaký ako pri BCI, pričom MAP-ms metóda je naviac rozšírená o status (napr. anomália áno/nie), ktorý je reprezentovaný distribúciou diskrétnej premennej každého merania senzoru.
\end{itemize}
\myFigure{images/2_anomalie_DBN}{Štruktúra dnamickej bayseovej siete. Vektor $X$ reprezentuje spojitú zložku, neznáme alebo tiež nazývané skryté premenné systému a vektory $M$ predstavujú spojité pozorované premenné v čase $t$.}{anomalie-dbn}{0.65}{h!}\label{fig:anomalie-dbn}


%%%%%%%%%%%%%%%%%%%%%%%%%%%%%%%%%%%%%%%%%%%%%%%%%
%                     FP                        %
%%%%%%%%%%%%%%%%%%%%%%%%%%%%%%%%%%%%%%%%%%%%%%%%%
%\section{FP} -- otazne


%%%%%%%%%%%%%%%%%%%%%%%%%%%%%%%%%%%%%%%%%%%%%%%%%
%                Zhlukovanie                    %
%%%%%%%%%%%%%%%%%%%%%%%%%%%%%%%%%%%%%%%%%%%%%%%%%
\section{Zhlukovanie}


%%%%%%%%%%%%%%%%%%%%%%%%%%%%%%%%%%%%%%%%%%%%%%%%%
%                Klasifikácia                   %
%%%%%%%%%%%%%%%%%%%%%%%%%%%%%%%%%%%%%%%%%%%%%%%%%
\section{Klasifikácia}


%%%%%%%%%%%%%%%%%%%%%%%%%%%%%%%%%%%%%%%%%%%%%%%%%
%                Zhlukovanie                    %
%%%%%%%%%%%%%%%%%%%%%%%%%%%%%%%%%%%%%%%%%%%%%%%%%
\section{Evaluácia}





\section{Detekcia trendov}
Detekcia trendov predstavuje kritickú úlohu pre analytikov. Reagovať na vzniknutý trend v čase jeho vzniku môže mať kritické dopady na fungovanie spoločnosti. Preto existuje záujem detekovať trendy v čase ich vzniku a byť schopný adekvátne reagovat príslušnými akciami v reálnom čase. Ak hovoríme o trendoch v obsahu, ktorý je generovaný používateľmi, napríklad na sociálnej sieti, potom sú trendy typicky poháňané udalosťami, ktoré náhle vznikajú a používatelia javia o ne záujem \citep{mathioudakis2010twittermonitor}. Mathioudakis and Koudas navrhli a implementovali metódu na detekciu trendov na sociálnej sieti Twitter\footnote{https://twitter.com/}. Metóda vykonáva detekciu trendov a ich následnú dodatočnú analýzu. Detekcia trendu pozostáva z dvoch krokov:
\begin{enumerate}
	\item \textit{detekcia nárazových kľúčových slov} identifikuje keď sa kľúčove slovo $K$ začne vyskytovať v prúde s neobvykle vysokým podielom v prúde. Napríklad náhly nárast frekvencie kľúčového slova \textit{NBA} môže byť spojený s prebiehajúcim dôležitým zápasom NBA. Pre detekciu nárazových kľúčových slov navrhli \citep{mathioudakis2010twittermonitor} nový algoritmus  \textit{QueueBurst} s nasledujúcimi charakteristikami:
		\begin{enumerate}
			\item \textit{jeden prechod} (angl. one-pass). Keďže ide o prúdové spracovanie, dáta môžu byť prečítané iba raz.
			\item \textit{spracovanie v reálnom čase}. Identifikácia nárazových kľúčových slov je vykonávané tak ako dáta vznikajú.
			\item \textit{odolnosť voči falošným nárazovým kľúčovým slovám}. Niekedy sa stane, že kľúčové slovo začne nárazovo prúdiť, ale nemusí predstavovať prúd, môže sa vyskytnúť zhodou okolností.
			\item \textit{odolnosť voči spam-u}. Existuje veľa automatických botov a používateľov, ktorí generujú spamujúce správy. Spam by mohol značne znížiť presnosť detekcie trendu.
		\end{enumerate}		
	\item \textit{zoskupovanie nárazových kľúrových slov} po tom čo algoritmus \textit{QueueBurst} identifikuje $\displaystyle K_t$ kľúčových slov pre každý časový moment $t$, sú kľúčové slová $\displaystyle k \in K_t$ periodicky zoskupované do nesúvislých (angl. disjoint) podmnožín $\displaystyle K_t^i \in K_t$. Potom identifikovaný trend predstavuje podmnožina $\displaystyle K_t^i$. Zoskupovanie vykonáva algoritmus \textit{GroupBurst}, ktorý posuďuje spoločný výskyt v posledných správach. Algoritmus je realizovaný lačnou stratégiou.
	\item Posledným krokom je analýza identifikovaného trendu $\displaystyle K_t^i$. Prvým krokom je identifikovať ďalšie kľúčové slová, ktoré sa spájajú s trendom $\displaystyle K_t^i$. Toto je dosiahnuté algoritmami na extrakciu kontextu, ktoré sú spustené na nedávnej histórií správ. Algoritmus vráti kľúčové slová, ktoré najviac korelujú s identifikovaným trendom $\displaystyle K_t^i$. Navyše, trendy na sociálnej sieti často pozostávajú z komentárov na aktuálne správy a novinky vo svete (napr. nytimes.com). Preto má zmysel ďalej extrahovať aj príslušné hypertextové odkazy a prideliť ich k trendu $\displaystyle K_t^i$. Posledný krok tejto metódy je zobrazenie priebehu identifikovaného trendu pomocou vizualizácie pre používateľa.
\end{enumerate}
\section{Rozpoznanie pocitu a nálady z používateľom generovaného obsahu}
Analýza pocitu alebo nálady (angl. sentiment analysis) može byť chápaný ako problém klasifikácie. Úlohoou je to klasifikovať správy (najčastejšie v kontexte sociálnych sietí) do dvoch kategórií na zákalde ich pozitívnych alebo negatívnych dojmov. Ak by sme pracovali s dátami zo sociálnej siete Twitter, je možné použiť na označkovanie správ. pomerne dobre, extrahovaním emotikonov, ktoré vyjadrujú pocity používateľa \citep{bifet2010sentiment}.  \par
Bifet and Frank publikovali v práci \citep{bifet2010sentiment} tri metódy a ich overenie na rozpoznanie nálady a pocitov z používateľom generovaného obsahu na sociálnej sieti Twitter. Experimentovali s tromi inkrementálnymi metódami, ktoré sú vhodné na spracovanie prúdu dát.

\paragraph{Multinomiálny Naive Bayes} je klasifikátor najčastejšie používaný na klasifikáciu dokumentov, ktorý obvykle poskytuje dobré výsledky aj čo sa týka presnosti výsledku aj rýchlosti. Túto metódu je jednoduché aplikovať v kontexte prúdu dát \citep{bifet2010sentiment}. Multinomiálny naivný Bayes sa pozerá na dokument ako na zhluk slov. Pre každú triedu $c$, $P(w|c)$, pravdepodobnosť, že slovo $w$ patrí do tejto triedy je odhadovaná z trénovacích dát jednoducho vypočítaním relatívnej početnosti každého slova v trénovacej sade pre danú triedu. Klasifikátor potrebuje naviac nepodmienenú pravdepodobnosť $P(c)$. Za predpokladu, že $\displaystyle n_{wd}$ je počet výskytov slova $w$ v dokumente $d$, pravdepodobnosť triedy $c$ z testovacieho dokumentu je nasledovaná: \newline
\begin{align*}
P(c|d) = \frac{P(c)\prod _{w \in d} P(w|c)^{n_{wd}}} {P(d)}
\end{align*}
Kde $P(d)$ je normalizačný faktor. Aby sme sa vyhli problému kedy sa trieda nevyskytuje v datasete ani jeden krát, je bežné použitie Laplacovej korekcie a nahradenie nulových početností jednotkou, resp. inicializovať početnosť každej triedy na 1 namiesto 0.

\paragraph{Stochastický gradientný zostup} (angl. Stochastic Gradient Descent, SGD). Bifet and Frank v ich práci použili implementáciu tzv. vanilla stochastický gradientný zostup s pevnou rýchlosťou učenia, optimalizujúc stratu s $L_2$ penalizáciou. $L_2$ penalizácia je často používaná pri podporných vektorových strojoch (angl. support vektor machines). Lineárny stroj, ktorý je často aplikovaný na problémy klasifikácie dokumentov, optimalizujeme funkciu straty nasledovne:
\begin{align*}
\frac{\lambda }{2}\left \| w \right \|^{2}+\sum [1-(yxw + b)]_{+}
\end{align*}
kde $w$ je váhovaný vektor, $b$ je sklon, $\lambda$ regulačný parameter a označenie triedy $y$ je z intervalu $\{+1, -1\}$.

\paragraph{Hoeffdingov strom} (angl. Hoeffding tree) je najznámejšia implementácia rozhodovacích stromov v použití prúdového spracovania. Hoeffdingov algoritmus implementuje stratégiu pred-prerezávania, ktorá je založená na Hoeffdingovom ohraničení. Toto umožňuje inkrementálne budovanie rozhodovacieho stromu. Uzol stromu je rozvinutý hneď ako obsahuje dostatočne silnú štatistickú informáciu. Existujú ďalšie sofistikované implementácie Hoeffdingových stromov, ktoré implementujú rýchlejšie a efektívnejšie algoritmy, napr. VFDT - Very Fast Decision Trees \citep{domingos2000mining} alebo CVFDT - Concept adapting Very Fast Decision Trees \citep{hulten2001mining}


\section{Zhodnotenie}
Existuje mnoho úloh, ktoré potrebujú často riešiť doménový experti. Pre tieto úlohy existujú celé sady algoritmov a metód, ktoré boli navrhnuté všeobecne alebo pre špecifickú doménu. Zaoberať sa všetkými úlohami pri spracovaní prúdu údajov je mimo rozsahu tejto práce a tiež rozobrať každú metódu pre analytickú úlohu.  \par
V tejto kapitole sme zhrnuli najčastejšie úlohy, s ktorými sa stretávajú doménovy experti a vybrané metódy na ich riešenie. Opísané metódy sú doménovo špecifické a nie sú použiteľné univerzálne pre každú doménu typu dát genericky.
\emptydoublepage
\chapter{Existujúce nástroje pre analýzu prúdu udalostí}
\label{Existujúce nástroje pre analýzu prúdu udalostí} 

\section{MOA}
Masívny online analyzátor (z angl. Massive Online Analysis - MOA, ďalej len MOA) je softvérové prostredie pre implementáciu algoritmov a vykonávanie experimentov pre online učenie sa z vyvýjajúcich sa prúdov dát \citep{DBLP:journals/jmlr/BifetHKP10}. MOA pozostáva z kolekcie offline a online metód a tiež nástrojov pre evaluáciu týchto metód. MOA implementuje metódy a algoritmy pre klasifikáciu, zhlukovanie prúdu, detekciu inštancií, ktoré sa vymykajú prahovým hodnotám a tiež odporúčacie systémy. Presnejšie MOA implementuje napríklad nasledujúce: stupňovanie (angl. boosting), vrecovanie (angl. bagging) a Hoeffdingove stromy, všetky metódy s a bez Naive Bayes klasifikátorom na listoch. MOA podporuje obojsmernú interakciu s nástrojom WEKA, ktorý je detailne opísaný v nasledujúcej kapitole. \\

MOA je implementovaná v programovacom jazyku Java. Za hlavný benefit implementácie v Jave považujú autori jej platformová nezávislosť. MOA obsahuje tiež generár prúdu dát, vie dobre modelovať concept drift. V aplikácií je možné definovať pravdepodobnosť, že inštancia prúdu patrí do nového concept drift-u. Sú dostupné nasledovné generátory prúdu \citep{DBLP:journals/jmlr/BifetHKP10}: \textit{Random Tree Generator, SEA Concepts Generator, STAGGER Concepts Generator, Rotating Hyperplane, Random RBF Generator, LED Generator, Waveform Generator, and Function Generator}.
\myFigure{images/3_moa}{Hlavná obrazovka GUI nástroja MOA.}{moa}{0.45}{h!}\label{fig:moa}

\section{WEKA}
The Wakaito Enviroment for Knowledge Analysis (ďalej len Weka) vznikol s jednoduchým cieľom poskytunúť výskumníkom unifikovanú platformu pre prístup k state-of-the-art technikám strojového učenia sa \citep{hall2009weka}. Weka vznikla na University of Waikato na Novom Zélande v roku 1992, pričom je aktívne vyvýjaná posledných 16 rokov. Weka poskytuje kolekciu algoritmov strojového učenia sa pre úlohy dolovania v dátach. Algoritmy môžu byť priamo aplikované na datasety prostredníctvom aplikácie alebo pouťžité vo vlastných aplikáciach volaním Java kódu. Weka obsahuje tieť nástroje na predspracovanie dát, klasifikáciu, regresiu, zhlukovanie, asociačné pravidlá a vizualizáciu. Nástroj je tieť vhodný pre návrhovanie a vývoj nových schém pre strojové učenia sa v kontexte dolovania dát. Zaujímavosťou je tiež, že Weka je nelietajúci vták, ktorý žije len na ostrove Nového Zélandu. \\

Vieľom nástroja je poskytnúť pracovný nástroj pre výskumníkov. Poskytuje napríklad (nástroj ich obsahuje omnoho viac, vymenované sú len vybrané) tieto algoritmy určené pre klasifikáciu dát:
\begin{itemize}
	\item \textit{Bayesová logistická regresia} (angl. Bayesian logistic regression), pre kategorizáciu textu s Gausovským a Laplacovým apriori.
	\item \textit{Najlepší prvý rozhodovací strom} (angl. Best-first decision tree), konštrukcia rozhodovacieho stromu so stratégiou najlepší prvý.
	\item \textit{Hybridná rozhodovacia tabuľka a naivný Bayes} (angl. Decision table naive Bayes hybrid) hybridný klasifikátor, ktorý kombinuje rozhodovacie tabuľky a metódu Naivný Bayes.
	\item \textit{Funkčné stromy} sú rozhodovacie stromy s lomeným rozdelením a lineárnymi funkciami v listoch.
\end{itemize}
Weka poskytuje tiež nástroje pre predspracovanie dát, zoznam niektorých filtrov (vymenované sú len vybrané základné filtre):
\begin{itemize}
	\item \textit{Pridanie klasifikátora}, pridá predikcie klasifikátora do datasetu.
	\item \textit{Pridanie ID} ako nového atribútu pre každý záznam datasetu.
	\item \textit{Pridanie hodnoty} chýbajúcim hodnotám z poskytnutého zoznamu.
	\item \textit{Preskupenie atribútov} preusporiadanie poradia atribútov.
	\item \textit{Numerické hodnoty na nominálne}, konverzia numerických hodnôt na nominálne.
\end{itemize}
\myFigure{images/3_weka}{Hlavná obrazovka GUI nástroja WEKA.}{weka}{0.5}{h!}\label{fig:weka}

\newpage
\section{RapidMiner Streams-Plugin}
Streams plugin poskytuje operátory RapidMiner-u pre základné budovanie blokov Streams API použítím obaľovača (angl. wrapper) na priame použitie implementácie, ktorú poskytuje Streams balík. Operátory Streams Plugin-u sú automaticky vytvorené pomocou procesora a použitím knižnice RapidMiner Beans \citep{bockermann2012processing}. Architektúra Streams Plugin-u je postavená na Streams API, ktoré bolo navrhnuté v práci Bockermanna a Bloma.
\myFigure{images/3_rapidminer}{Architektúra RapidMiner Stream Plugin-u a ďalších potrebných častí.}{rapidminer}{0.65}{h!}\label{fig:rapidminer}

\section{StreamBase}
StreamBase\footnote{http://www.streambase.com/} je platforma pre spracovanie udalostí, ktorá poskytuje vysoko-výkonný softvér pre budovanie a nasadanie systémov, ktoré analyzujú a reagujú (napr. akciami) na prúdiace dáta v reálnom čase. StreamBase poskytuje prostredie pre svižný vývoj, server pre spracovanie udalostí s nízkou odozvou a vysokou priepustnosťou a zároveň integráciu do podnikových nástrojov, napríklad pre spracovanie historických údajov. Server analyzuje prúdiace dáta a poskytuje výsledky a odpovede v reálnom čase s extrémne nízkou odozvou. Toto je dosiahnuté maximalizáciou využitia hlavnej pamäte a ostatných prostriedkov servera, zatiaľ čo sa eliminujú závislosti na ostatné aplikácie. Integrované vývojové prostredie - StreamBase Studio umožňuje programátorom jednoducho a rýchlo vytvoriť, testovat a debugovať StreamSQL aplikácie použitím grafického modelu toku vykonávania. StreamBase aplikácie sú potom skompilované a nasadené za behu servera. \\

StreamSQL je dopytovací jazyk, ktorý rozširuje štandard SQL. StreamSQL umožňuje spracovanie prúdov v reálnom čase a dopytovanie sa do nich. Základná myšlienka jazyka SQL je možnosť dopytovať sa do uložených statických kolekcií dát, StreamSQL umožňuje to isté, ale do prúdov dát. Teda, StreamSQL musí zvládnuť spracovať kontinuálny prúd udalostí a časovo orientované záznamy. StreamSQL zachováva schopnosti jazyka SQL zatiaľ čo pridáva nové možnosti ako napríklad: bohatý systém posuvných okien, možnosť miešania prúdiacich dát a statických dát a tiež možnosť pridať vlastnú logiku vo forme analytických funkcií. \\

StreamBase EventFlow je jazyk pre prúdove spracovanie vo forme tokov a operátorov ako grafických elementov. Používateľ má možnosť spájať tieto grafické elementy a vytvárať tak jednoducho topológiu pre prúdové spracovanie bez nutnosti programovania. EventFlow integruje všetky možnosti StreamSQL. \\

Použitie StreamBase sa výborne hodí pre štrukturované aplikácie "reálneho času", ktoré majú za cieľ rýchle spracovanie spolu s rýchlym prototypovaním a nasadením nových funkcionalít.
\myFigure{images/3_streambase}{Vývojové prostredie nástroja StreamBase.}{streambase}{0.55}{h!}\label{fig:streambase}

\section{InforSphere Streams (IBM)}




















\emptydoublepage
%\include{implementation}
\chapter{Metóda pre detekciu trendov v prúde udalostí}
\label{Metóda pre detekciu trendov v prúde udalostí}
Navrhujeme metódu detekovanie trendov v prúde udalostí s polo-automatickým výberom vhodnej metódy. Našim cieľom je odbremeniť doménového experta od náročného výberu správnej metódy pre detekciu trendu v prúde. Mohli by sme sa zamerať na detekciu trendov v doménovo špecifickej oblasti, ale tento problém je z veľkej miery preskúmaný. Preto sa zameriavame na rôzne domény, či ide o senzorové syntetické dáta, dáta z energetickej alebo počítačovej siete, či používateľmi vytváraný obsah na sociálnych sietiach a mikro-blogových službách. Cieľom je teda: použiteľnosť a jednoduchosť nami navrhovanej metódy pre doménového experta bez nutnosti detailnej znalosti o fungovaní metódy a modelu, ktoré vytvára.
Teda, kladieme nasledujúce hypotézy:
\begin{hypothesis}{Naše riešenie detekuje trendy v prúde dát s istou pravdepodobnosťou použitím polo-automatického výberu vhodnej metódy pre detekciu trendov a zároveň poskytuje výsledky v reálnom čase}
\end{hypothesis}
\begin{hypothesis}{Metóda je ľahká na použitie a interpretované výsledky sú jednoduché na pochopenie pre doménového experta bez detailnej znalosti o fungovaní modelu.}
\end{hypothesis}

Pri detekcií trendu bude nutné zohľadniť nasledujúce elementy:
\begin{itemize}
	\item anomálie, chyby a spam,
	\item sezónnosť dát,
	\item concept drift a zmeny,
\end{itemize}
\emptydoublepage
\chapter{Zhodnotenie a budúca práca}
\label{Zhodnotenie a budúca práca}


\emptydoublepage

%--------------------------------------------------------------
% prilohy, index
%--------------------------------------------------------------
\backmatter

%bibliography
\clearpage
\phantomsection
\addcontentsline{toc}{chapter}{\protect\numberline{}Literatúra}
\renewcommand{\bibname}{Literatúra}
\bibliography{thesis}
\emptydoublepage

\begin{appendix}
\label{app.01}
\appendix
\chapter*{Prílohy}
\addcontentsline{toc}{chapter}{Prílohy}
\renewcommand{\thesection}{\Alph{section}}


% ======== DOKUMENTACIA ======= %
\section{Plán na letný semester 2016/2017}\label{plan-leto}
Tento plán popisuje náš plán ďalšieho vývoja diplomovej práce v nasledujúcom letnom semestri na týždennej granularite. Pričom predpokladáme, že semester má 12 týždňov.
\begin{itemize}
	\item \textit{1-2 týžden}: Príprava článku na študentskú vedeckú konferenciu IIT.SRC.
	\item \textit{3 týždeň}: Vyhodnotenie prezentovaných výsledkov na IIT.SRC, určenie ďalšieho smeru a priestoru na zlepšenie aktuálneho stavu implementovanej metódy.
	\item \textit{4. týždeň}: Implementácia návrhov na zlepšenie metódy pre klasifikáciu a ich vyhodnotenie.
	\item \textit{5. týždeň}: Implementácia návrhov na zlepšenie metódy pre vizualizáciu a ich vyhodnotenie. 
	\item \textit{6. týždeň}: Vyhodnotenie kvality a výkonnosti implementovanej metódy - kvantitatívne vyhodnotenie stanovených metrík kvality.
	\item \textit{7. týždeň}: Návrh ďalších experimentov vo forme používateľskej a expertnej štúdie v použiteľnosti navrhovanej metódy.
	\item \textit{8. týždeň}: Používateľská štúdia a spísanie výsledkov kvantitatívných metrík kvality metódy z 5-6. týždňa.
	\item \textit{9. týždeň}: Vyhodnotenie používateľskej štúdie.
	\item \textit{10. týždeň}: Analýza priestoru na zlepšenie navrhovanej a implementovanej metódy podľa výsledkov používateľskej štúdie.
	\item \textit{11. týždeň}: Spísanie výsledkov používateľskej štúdie a finalizácia diplomovej práce, príprava na odovzdanie.
	\item \textit{12. týždeň}: Odovzdanie diplomovej práce.
\end{itemize} 

\section{Plán na zimný semester 2016/2017}\label{plan-zima}
\begin{itemize}
	\item Rozšírenie analýzy o ďalšie metódy a celkovo zpresnenie, zprehľadnenie a orezanie analýzy.
	\item Dokončenie návhu vlastnej metódy.
	\item Implementácia metódy.
	\item Evaluácia metódy - toto je priamo súčasťou navrhovanej metódy.
	\item Príprava článku na nejakú konferenciu, napr. IIT.SRC.
	\item Prvý experiment s použitím Eye Tracker-a na evaluáciu vizualizácie výsledkov.
\end{itemize}


%\include{appendix_02}
%\include{appendix_03}
%\include{appendix_04}
\emptydoublepage
%\include{appendix_02}
%\emptydoublepage
\end{appendix}


%index
%\markboth{Index}{Index}
%\thispagestyle{plain}
%\phantomsection
%\addcontentsline{toc}{chapter}{\protect\numberline{}Index}
%\printindex

%date of typesetting
\newpage
\thispagestyle{empty}
\vspace*{\fill}
%\hspace*{\fill}{\tiny Typeset \today}

%--------------------------------------------------------------
\end{document}

