\chapter{Klasifikácia prúdu dát použitím rozhodovacích stromov}
\label{Klasifikácia prúdu dát použitím rozhodovacích stromov}
% TODO: Uvod


Teda, kladieme nasledujúce hypotézy:
\begin{hypothesis}{Naše riešenie detekuje trendy v prúde dát s istou pravdepodobnosťou použitím polo-automatického výberu vhodnej metódy pre detekciu trendov a zároveň poskytuje výsledky v reálnom čase}
\end{hypothesis}
\begin{hypothesis}{Metóda je ľahká na použitie a interpretované výsledky sú jednoduché na pochopenie pre doménového experta bez detailnej znalosti o fungovaní modelu.}
\end{hypothesis}

Pri detekcií trendu bude nutné zohľadniť nasledujúce elementy:
\begin{itemize}
	\item anomálie, chyby a spam,
	\item sezónnosť dát,
	\item concept drift a zmeny,
\end{itemize}

V situácií keď potrebuje doménovy expert analyzovať trend v dátach reálneho sveta, je často potrebná najprv detailná znalosť dát, ktorú doménový expert, predpokladáme má. Potom, pre to, aby vedel správne analyzovať trend, sa potrebuje oboznámiť s viacerými modelmi a následne ich vyhodnotiť. Práve túto činnosť chceme uľahčiť doménovým expertom našou metódou. Nami navrhovaná metóda, schematicky zobrazená na \ref{fig:method-proposal} pozostáva z nasledujúcich krokov:
\begin{enumerate}
	\item Používateľ vyberie metódu, ktorú chce aplikovať na analýzu trendu v prúde dát. Používateľ môže metódu vybrať sám, alebo nechá aplikáciu aby mu odporučila podľa vzorky dát vhodnú metódu.
	\item Paralelne sa nepretržite spracuje a analyzuje prúd údajov. Zároveň je vytváraný model.
	\item Po každej spracovanej správe je model vyhodnotený viacerými metrikami. Podľa výsledku evaluácie sa upraví aktuálny model.
	\item Posledným krokom je prezentácia výsledkov používateľovi vo forme vizualizácie.
\end{enumerate}
\myFigure{images/DP_PeWe_method-propose}{Schematický návrh metódy pre semi-automatickú analýzu trendov.}{method-proposal}{0.45}{h!}\label{fig:method-proposal}


