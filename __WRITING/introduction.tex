%!TEX root = ./main.tex
%
% This file is part of the i10 thesis template developed and used by the
% Media Computing Group at RWTH Aachen University.
% The current version of this template can be obtained at
% <http://www.media.informatik.rwth-aachen.de/karrer.html>.

\chapter{Úvod}
%\label{introduction}
V súčasnosti pozorujeme vysoký nárast záujmu v oblasti analýzy údajov. Vhodné použitie metód a techník na analýzu prináša hodnotné výstupy pre používateľa, ktoré môžu môžu byť použité pre strategické rozhodnutia v podnikoch. Najčastejší postup je aplikovaním metód ako napríklad lieviková analýza alebo rozhodovacie stromy nad statickou kolekciu dát. Avšak tento prístup neposkytuje okamžitý výsledok pre používateľa. Nedostatkom takýchto prístupov je nutnosť dáta najskôr zozbierať a uložiť, čo je dnes, kedy vznikajú milióny záznamov za deň, veľký problém.\\
\\
Pod pojmom spracovanie v reálnom čase myslíme spracovanie v takmer reálnom čase, tzv. mäkké (angl. soft) spracovanie v reálnom čase. Presné limity, do kedy sa spracovanie považuje za reálny čas závisí od problému. Niekde to môže predstavovat rádovo stotiny sekundy, inde rádovo sekundy. \\
\\
Pri dolovaní v prúde dát čelíme niekoľkým výzvam: objem,  rýchlosť (frekvencia) a rozmanitosť. Veľký objem dát, ktoré vznikajú veľmi rýchlo je potrebné spracovať v ohraničenom časovom intervale, často v takmer reálnom čase (závisí od kontextu problému). Objem dát sa neustále zvyšuje, potenciálne až do nekonečna. Identifikujeme niekoľko najviac zasiahnutých oblastí, ktoré sú zdrojmi týchto dát, a to senzory, počítačové siete, sociálne siete a Internet Vecí (angl. Internet of Things). Na informácie generované z takýchto zdrojov sa často pozeráme ako na neohraničené a potenciálne nekonečné prúdy údajov. \\
\\
Spracovanie a nasledujúca analýza týchto prúdov je komplexná úloha. Pre aplikácie je kritické spracovať údaje s nízkou odozvou, pričom riešenie musú byť presné, škálovateľné a odolné voči chybám. Nakoľko sú prúdy neohraničené vo veľkosti a potenciálne nekonečné, môžeme spracovať len ohraničený interval prúdu. Potom hovoríme, že dáta musia byť spracované tak ako vznikajú. Tradičné metódy a princípy pre spracovanie statickej kolekcii údajov nie sú postačujúce na takéto úlohy. 