%!TEX root = ./main.tex

\chapter{Úvod}
%\label{introduction}
V súčasnosti pozorujeme zvýšený záujem o oblasť analýzy a dolovania dát. Vhodné použitie a výber metód pre spracovanie dát prináša hodnotné výstupy a náhľady pre používateľa. Výstupy môžu byť použité pre strategické rozhodnutia v podnikoch. Najčastejší postup je aplikovaním metód ako napríklad lieviková analýza alebo rozhodovacie stromy nad statickou kolekciu dát. Tento prístup má niekoľko problémov a to najmä: všetky trénovacie dáta musia byť uložené v pamäti alebo na disku, spracovanie a výpočtová náročnosť a vysporiadanie sa s trendami a zmenami v dátach. Nutnosť dáta najskôr zozbierať a uložiť, čo je dnes, kedy vznikajú milióny záznamov za deň, či hodinu, predstavuje rovnako veľký problém.\\
\\
Pod pojmom spracovanie v reálnom čase myslíme spracovanie v takmer reálnom čase, tzv. jemné (angl. soft) spracovanie v reálnom čase. Jemné spracovanie v reálnom čase znamená, že systém negarantuje spracovanie a odpoveď v stanovenom časovom limite, pričom niektoré vzorky sa môžu omeškať alebo úplne vynechať \citep{stankovic1988real}. Presné limity, do kedy sa spracovanie považuje za reálny čas závisí od problému. Niekde to môže predstavovat rádovo stotiny sekundy, v inej úlohe rádovo sekundy. V tejto práci budeme pracovať s pojmom spracovanie v reálnom čase chápajúch ako jemné spracovanie v reálnom čase.\\
\\
Pri dolovaní v prúde dát čelíme niekoľkým výzvam: objem,  rýchlosť (frekvencia) a rozmanitosť. Veľký objem dát, ktoré vznikajú veľmi rýchlo je potrebné spracovať v ohraničenom časovom intervale, často v reálnom čase. Pričom sa objem dát neustále zväčšuje, potenciálne narastá až do nekonečna. Identifikujeme niekoľko najviac zasiahnutých oblastí, ktoré sú zdrojmi týchto dát: počítačové siete, sociálne siete, Webové stránky (sledovanie správania používateľa na stránke) a Internet Vecí (angl. Internet of Things). Na informácie generované z takýchto zdrojov sa často pozeráme ako na neohraničené a potenciálne nekonečné prúdy údajov.\\
\\
Spracovanie, analýza a dolovanie v týchto prúdoch je komplexná úloha. Pre aplikácie je kritické spracovať údaje s nízkou odozvou, pričom riešenie musí byť presné, škálovateľné a odolné voči chybám. Nakoľko sú prúdy neohraničené vo veľkosti a potenciálne nekonečné, môžeme spracovať len ohraničený interval prúdu. Potom hovoríme, že dáta musia byť spracované tak ako vznikajú. Tradičné metódy a princípy pre spracovanie statickej kolekcii údajov nie sú postačujúce na takéto úlohy \citep{krempl2014open, han2011data}.