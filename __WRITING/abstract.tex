\loadgeometry{myAbstract}
\chapter*{Anotácia\markboth{Abstrakt}{Abstrakt}}
%\addcontentsline{toc}{chapter}{\protect\numberline{}Abstrakt}
\label{abstrakt}
\begin{center}
\textbf{Fakulta Informatiky a Informačných Technológií}\\
\textbf{Slovenská Technická Univerzita}\\
\end{center}
\begin{tabular}{ p{15em} p{15em} }
Meno: & Bc. Matúš Cimerman\\
Vedúci diplomovej práce: & Ing. Jakub Ševcech\\
Diplomová práca: & Analýza prúdu prichádzajúcich udalostí použitím rôznych metód pre analýzu údajov\\
Študijný program: & Informačné systémy\\
Máj 2016
\end{tabular}

%here (slovak version)
Dnes môžeme pozorvať narastujúcu potrebu analyzovať dáta počas ich vzniku. Spracovanie a analýza prúdov dát predstavuje komplexnú úlohu, pričom je dôležité poskytnúť riešenie s nízkou odozvou, ktoré je odolné voči chybám.

V našej práci sa sústreďujeme na návrh súboru nástrojov, ktoré pomôžu doménovému expertovi počas analýzy dát. Doménový expert nepotrebuje mať detailné znalosti o fungovaní analytického modelu. Podobný prístup je podobný, ak chceme analyzovať statické kolekcia dát napríklad lievikovou analýzou. Študujeme možnosti použitia tradičných metód pre statické údaje v doméne analýzy prúdov dát. Našim cieľom je aplikovať metódu pre analýzu v doméne prúdu dát. 
Navrhujeme použitie rozhodovacieho stromu v kontexte klasifikácie prúdu dát, ktorý používa Hoeffdingovu mieru pre výber najlepšieho rozhodovacieho atribútu so stanovenou istotou. Pre vysporiadanie so zmenami aplikujeme algoritmus ADWIN, ktorý adaptívne dokáže detekovať zmeny v prúde. 
Zameriavame sa pritom na jednoduchosť vybranej metódy a interpretovateľnosť výsledkov. Pre doménových expertov je nevyhnutné aby boli tieto požiadavky splnené, pretože nebudú potrebovať detailné znalosti z domén ako strojové učenie sa alebo štatistika. Naše riešenie vyhodnocujeme implementovaním softvérovej súčiastky a vybranej metódy.


%=== ENGLISH VERSION===% 
\emptydoublepage
\chapter*{Annotation\markboth{Abstract}{Abstract}}
%\addcontentsline{toc}{chapter}{\protect\numberline{}Abstract}
\label{abstract}
\begin{center}
\textbf{Faculty of Informatics and Information Technologies}\\
\textbf{Slovak University of Technology}\\
\end{center}
\begin{tabular}{ p{10em} p{15em} }
Name: & Bc. Matúš Cimerman\\
Supervisor: & Ing. Jakub Ševcech\\
Diploma thesis: & Stream analysis of incoming events using different data analysis methods\\
Course: & Information systems\\
2016, May
\end{tabular}

%here (english version)
Nowadays we can see emerging need for data analysis as data occur. Processing and analysis of data streams is a complex task, first, we particuraly need to provide low latency and fault-tolerant solution.

In our work we focus on proposal a set of tools which will help domain expert in process of data analysis. Domain expert do not need to have detailed knowledge of analytics models. Similar approach is popular when we want analyse static collections, eg. funnel analysis. We study possibilities of usage well known methods for static data analysis in domain data streams analysis. Our goal is to apply method for data analysis in domain of data streams. 
We propose usage of decision tree in context of data stream classification, which uses Hoeffding bound to select best splitting attribute with desired confidence. To adaptively deal with concept drift we are using algorithm ADWIN to detect drifts in stream. 
This approach is focused on simplicity in use of selected method and interpretability of results. It is essential for domain experts to meet these requirements because they will not need to have detailed knowledge from such a domains as machine learning or statistics. We evaluate our solution using software component implementing chosen method.

\emptydoublepage
\loadgeometry{myText}
