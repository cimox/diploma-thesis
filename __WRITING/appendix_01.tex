\label{app.01}
\appendix
\chapter*{Prílohy}
\addcontentsline{toc}{chapter}{Prílohy}
\renewcommand{\thesection}{\Alph{section}}


% ======== DOKUMENTACIA ======= %
\section{Plán na letný semester 2016/2017}\label{plan-leto}
Tento plán popisuje náš plán ďalšieho vývoja diplomovej práce v nasledujúcom letnom semestri na týždennej granularite. Pričom predpokladáme, že semester má 12 týždňov.
\begin{itemize}
	\item \textit{1-2 týžden}: Príprava článku na študentskú vedeckú konferenciu IIT.SRC.
	\item \textit{3 týždeň}: Vyhodnotenie prezentovaných výsledkov na IIT.SRC, určenie ďalšieho smeru a priestoru na zlepšenie aktuálneho stavu implementovanej metódy.
	\item \textit{4. týždeň}: Implementácia návrhov na zlepšenie metódy pre klasifikáciu a ich vyhodnotenie.
	\item \textit{5. týždeň}: Implementácia návrhov na zlepšenie metódy pre vizualizáciu a ich vyhodnotenie. 
	\item \textit{6. týždeň}: Vyhodnotenie kvality a výkonnosti implementovanej metódy - kvantitatívne vyhodnotenie stanovených metrík kvality.
	\item \textit{7. týždeň}: Návrh ďalších experimentov vo forme používateľskej a expertnej štúdie v použiteľnosti navrhovanej metódy.
	\item \textit{8. týždeň}: Používateľská štúdia a spísanie výsledkov kvantitatívných metrík kvality metódy z 5-6. týždňa.
	\item \textit{9. týždeň}: Vyhodnotenie používateľskej štúdie.
	\item \textit{10. týždeň}: Analýza priestoru na zlepšenie navrhovanej a implementovanej metódy podľa výsledkov používateľskej štúdie.
	\item \textit{11. týždeň}: Spísanie výsledkov používateľskej štúdie a finalizácia diplomovej práce, príprava na odovzdanie.
	\item \textit{12. týždeň}: Odovzdanie diplomovej práce.
\end{itemize} 

\section{Plán na zimný semester 2016/2017}\label{plan-zima}
\begin{itemize}
	\item Rozšírenie analýzy o ďalšie metódy a celkovo zpresnenie, zprehľadnenie a orezanie analýzy.
	\item Dokončenie návhu vlastnej metódy.
	\item Implementácia metódy.
	\item Evaluácia metódy - toto je priamo súčasťou navrhovanej metódy.
	\item Príprava článku na nejakú konferenciu, napr. IIT.SRC.
	\item Prvý experiment s použitím Eye Tracker-a na evaluáciu vizualizácie výsledkov.
\end{itemize}

