\chapter{Analytické úlohy v kontexte prúdu údajov}
\label{Analytické úlohy v kontexte prúdu údajov}
\textbf{TODO:} Asi bolo vhodné pridať nejaký úvod tejto kapitoly - zamyslieť sa, či to má zmysel.

\section{Predspracovanie prúdu}
Predspracovanie je azda najdôležitejším krokom v aplikáciach reálneho sveta a časovo najnáročnejšou úlohou pre každého analytika. Nakoľko dáta prichádzajú z nehomogénneho sveta, môžu byť zašumené, nekompletné, duplicitné alebo často obsahovať hodnoty, ktoré sa značne líšia od ostatných. Predspracovanie prúdiach údajov je potrebné čo najviac automatizovať. Existuje potreba pre implementovanie metód  strojového učenia, ktoré sa adaptujú v čase s meniacimi sa dátami. Avšak, tieto modely a metódy by mali byť synchronizované s prediktívnymi modelmi. Pri aplikovaní strojového učenia a prediktívnych modelov je nutnosť uvažovať historické dáta. Čím podstatne narastá zložitosť celého prístupu. Hlavné výskumné problémy rozdeľujeme do dvoch hlavých kategórií, \textit{dátovo orientované} a \textit{orientované na úlohu}. Dátovo orientované techniky používané pri spracovaní prúdiacich údajov: \newline
\begin{itemize}
	\item Vzorkovanie je proces výberu dátovej vzorky na spracovanie podľa pravdepodobnostného modelu. Problém pri vzorkovaní je v kontexte analýzy prúdiach dát je potenciálne nekonečný dataset, resp. nevedomosť jeho skutočnej veľkosti.
	\item Kontrolovanie zaťaženia je proces zahadzovania niektorých, potenciálne nepotrebných vzoriek dát. Opäť tu nástava problém v kontexte prúdu. Môže nastať moment, kedy práve zahodená vzorka dát je pre aktuálnu analýzu príznačná a dôležitá.
	\item Agregácia je proces vypočtu štatistických údajov nad prúdmi. Problém je pri veľkých tokoch v prúde, pričom je tiež potreba pohľadu do minulosti.
\end{itemize}
\begin{itemize}
	\item Aproximačné algoritmy použitie aproximačných algoritmov má za následok podstatné zrýchlenie spracovania a analýzy prúdov za predpokladu istej chybovosti. Chybovosť je zvačsa ohraničená.
	\item Posuvné okno, tento prístup vznikol s potrebou analýzy definovaného časového okna z prúdiacih údajov.
\end{itemize}

\section{Dolovanie a extrakcia informácií} 
Dolovanie z prúdiacih dát dnes predstavuje niekoľko výziev. Jednou z oblastí je bezpečnosť a dôvernosť v dolovaní dát. Návrh modelu môže byť navrhovaný na pôvodných dátach, ale nasadenie by malo byť realizované na anonymizovaných dátach. Identifikujeme dve hlavné výzvy pre uchovanie bezpečnosti pri dolovaní v prúdoch. Prvou je nekompletnosť informácií, informácie často prichádzajú nekompletné alebo neaktuálne. Druhá výzva nadväzuje na prvú, a to že dáta sa môžu v čase meniť a vyvýjať. Môže sa meniť štruktúra, komplexita a ich presnosť. Preto, vopred definované bezpečnostné pravidlá a politiky nemusia byť časom aktuálne a pokrývať stanovenú oblasť. Medzi známe techniky dolovania v prúdoch údajov považujeme nasledujúce:
\\
\begin{itemize}
	\item Klastrovanie, existuje niekoľko výskumov, ktoré sa venovali špeciálne klastrovaniu implementovaním napríklad k-mediánu a inkrementálnych algoritmov.
	\item Klasifikácia, opäť existuje niekoľko známych výskumov, ktoré s venujú problému klasifikácie. Napríklad použitím dát z reálneho sveta a umelých dát, pričom implementujú algoritmy, ktoré triedia dáta na základe porovnaní medzi týmito dvoma vzorkami.
	\item Počítanie frekvencie a opakovaní, použitím posuvných okien a inkrementálnych algoritmov na detekciu vzorov v prúde.
	\item Analýza časových radov použitím symbolickej reprezentácie šasových radov v prúde dát. Takáto reprezentácia nám umožňuje redukciu veľkosti prenášaných dát. Táto technika pozostáva z dvoch hlavných krokov, aproximácia po častiach a následná transformácia výsledku do diskrétnych veličín.
\end{itemize}

\section{Zhodnotenie}