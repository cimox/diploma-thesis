\chapter{Analytické úlohy v prúde udalostí}
\label{Analytické úlohy v prúde udalostí}
Spracovanie a dolovanie znalostí predstavuje výzvu, zvláštnu pozornosti si tieto úlohy vyžadujú pri spracovaní a analýze prúdu udalostí. Prúd udalostí je tiež často nazývaný \textit{prúd dát} alebo \textit{údajov}, či len skrátene \textit{prúd}. V tomto texte budeme pre jednoduchosť používať najmä termín \textit{prúd} \citep{tran2014change}. Avšak môžu sa vyskytnúť aj terminý ako: \textit{prúd dát}, \textit{prúd udalostí}, \textit{sekvencia udalostí, či elementov}, pričom všetky termíny majú v tomto texte \textit{rovnaký} význam.\\
\begin{definition}{Prúd je nekonečná sekvencia elementov}
$S$ = {()}
\end{definition}

Takmer každé odvetvie dnes generuje masívne množstvo dát, ktoré obsahujú hodnotné znalosti a poznanie. Vzhľadom na veľký objem vzniknutých dát, analytyci často strácajú schopnosť dolovať v celej sade dát. Stáva sa preto častým zvykom, že sa analyzuje len reprezentujúca vzorka, pretože to predstavuje menšiu časovú výzvu pre doménového experta \citep{hulten2001mining}. Tvrdíme, že bude pre doménoveho experta vysokým prínosom možnosť vykonávať analýzy nad prúdom v reálnom čase. Výstupy z takejto analýzy sú na rôznej granularite a úrovni, pričom môžu byť neskôr použité na ďalšie spracovanie alebo na priame prezentovanie výsledkov.

\section{Predspracovanie prúdu}
Predspracovanie je azda najdôležitejším krokom v aplikáciach reálneho sveta a časovo najnáročnejšou úlohou pre každého analytika. Nakoľko dáta prichádzajú z nehomogénneho sveta, môžu byť zašumené, nekompletné, duplicitné alebo často obsahovať hodnoty, ktoré sa značne líšia od ostatných. Predspracovanie prúdiach údajov je potrebné čo najviac automatizovať. Existuje potreba pre implementovanie metód  strojového učenia, ktoré sa adaptujú v čase s meniacimi sa dátami. Avšak, tieto modely a metódy by mali byť synchronizované s prediktívnymi modelmi. Pri aplikovaní strojového učenia a prediktívnych modelov je nutnosť uvažovať historické dáta. Čím podstatne narastá zložitosť celého prístupu. Hlavné výskumné problémy rozdeľujeme do dvoch hlavých kategórií, \textit{dátovo orientované} a \textit{orientované na úlohu}. Dátovo orientované techniky používané pri spracovaní prúdiacich údajov: \newline
\begin{itemize}
	\item Vzorkovanie je proces výberu dátovej vzorky na spracovanie podľa pravdepodobnostného modelu. Problém pri vzorkovaní je v kontexte analýzy prúdiach dát je potenciálne nekonečný dataset, resp. nevedomosť jeho skutočnej veľkosti.
	\item Kontrolovanie zaťaženia je proces zahadzovania niektorých, potenciálne nepotrebných vzoriek dát. Opäť tu nástava problém v kontexte prúdu. Môže nastať moment, kedy práve zahodená vzorka dát je pre aktuálnu analýzu príznačná a dôležitá.
	\item Agregácia je proces vypočtu štatistických údajov nad prúdmi. Problém je pri veľkých tokoch v prúde, pričom je tiež potreba pohľadu do minulosti.
\end{itemize}
\begin{itemize}
	\item Aproximačné algoritmy použitie aproximačných algoritmov má za následok podstatné zrýchlenie spracovania a analýzy prúdov za predpokladu istej chybovosti. Chybovosť je zvačsa ohraničená.
	\item Posuvné okno, tento prístup vznikol s potrebou analýzy definovaného časového okna z prúdiacih údajov.
\end{itemize}

\section{Dolovanie a extrakcia informácií} 
Dolovanie z prúdiacih dát dnes predstavuje niekoľko výziev. Jednou z oblastí je bezpečnosť a dôvernosť v dolovaní dát. Návrh modelu môže byť navrhovaný na pôvodných dátach, ale nasadenie by malo byť realizované na anonymizovaných dátach. Identifikujeme dve hlavné výzvy pre uchovanie bezpečnosti pri dolovaní v prúdoch. Prvou je nekompletnosť informácií, informácie často prichádzajú nekompletné alebo neaktuálne. Druhá výzva nadväzuje na prvú, a to že dáta sa môžu v čase meniť a vyvýjať. Môže sa meniť štruktúra, komplexita a ich presnosť. Preto, vopred definované bezpečnostné pravidlá a politiky nemusia byť časom aktuálne a pokrývať stanovenú oblasť. Medzi známe techniky dolovania v prúdoch údajov považujeme nasledujúce:
\\
\begin{itemize}
	\item Klastrovanie, existuje niekoľko výskumov, ktoré sa venovali špeciálne klastrovaniu implementovaním napríklad k-mediánu a inkrementálnych algoritmov.
	\item Klasifikácia, opäť existuje niekoľko známych výskumov, ktoré s venujú problému klasifikácie. Napríklad použitím dát z reálneho sveta a umelých dát, pričom implementujú algoritmy, ktoré triedia dáta na základe porovnaní medzi týmito dvoma vzorkami.
	\item Počítanie frekvencie a opakovaní, použitím posuvných okien a inkrementálnych algoritmov na detekciu vzorov v prúde.
	\item Analýza časových radov použitím symbolickej reprezentácie časových radov v prúde dát. Takáto reprezentácia nám umožňuje redukciu veľkosti prenášaných dát. Táto technika pozostáva z dvoch hlavných krokov, aproximácia po častiach a následná transformácia výsledku do diskrétnych veličín.
\end{itemize}

\section{Detekcia zmien}
Detekcia zmien zohráva, v dnešnom rýchlo sa meniacom svete, dôležitú úlohu. Zmeny nastávajú veľmi rýchlo a nečakane. Preto je preto stúpa potreba detekcie zmeny a následná správna reakcia, ktorá vyplynie z detekovanej zmeny. Na to aby sme boli schopný na tieto zmeny adekvátne reagovať je potrebné dáta spracovávať tak ako vznikajú a pozerať sa na ne ako na prúd udalostí. Tradičné metódy pre paralelné spracovanie uvažujú len statickú kolekciu dát \citep{tran2014change}.\\

Detekcia zmeny predstavuje proces identifikácie zmeny aktuálneho stavu objektu voči predchádzajúcemu. Na tento objekt sa pozeráme v rôznom čase. Dôležitý rozdiel medzi zmenou a rozdielom je, že zmena hovorí o prechode objektu do iného stavu, zatiaľ čo rozdiel znamená nepodobnosť v atribútoch dvoch objektov. V kontexte prúdu detekovanie zmeny je proces segmentácie prúdu udalostí do rôznych segmentov a identifikovanie miest kde sa zmení dynamika prúdu \citep{ross2009online}. Metóda pre detekciu zmien musí riešiť nasledujúce úlohy \citep{tran2014change}: \textit{detekcia zmeny} a \textit{lokalizácia zmeny}. Ďalší rozdiel, ktorý je potrebné zadefinovať, je medzi rozdiel detekovaním posunu pojmu (angl. concept drift, ďalej len \textit{concept drift}). Detekcia concept drift-u sa sústreďuje na označkované dáta, zatiaľ čo detekcia zmeny pracuje s označkovanými rovnako ako s neoznačkovanými dátami.\\

Metódy pre detekovanie zmien môžme klasifikovať do nasledujúcich prístupov \citep{liu2010mining}: \textit{metódy založené na stave}, \textit{metódy sledujúce trend} a \textit{prahové metódy}. Algoritmus pre detekciu zmien by mal spĺňať aspoň nasledovné požiadavky: \textit{presnosť}, \textit{rýchlosť} a \textit{odpoveď v reálnom čase}. Algoritmus by tiež mal detekovať čo najmenej chybných zmien a čo najviac správnych presných miest zmeny. Algoritmy by mali byť prispôsobené reálnemu prostrediu a spracovaniu prúdov vysokých objemov a rýchlostí. Na obrázku \ref{fig:zmeny-vseobecny-dia} je zobrazený všeobecný diagram pre detekciu zmeny v prúde udalostí.

\myFigure{images/2_zmeny_vseobecny-diagram}{Všeobecný diagram zobrazujúci detekciu zmeny v prúde udalostí.}{test-dia}{0.5}{h!}\label{fig:zmeny-vseobecny-dia}

Prístupy k detekovaniu zmien v prúde dát \citep{tran2014change}:
\begin{itemize}
	\item \textit{model prúdu dát} môže byť jeden z nasledujúcich: model časových radov, pokladničný model a model turniketu. Podľa modelu prúdu dáť existujú príslušné algoritmy, ktoré boli vytvorené pre daný model.
	\item \textit{charakteristika dát}, metódy pre detekciu zmien môžu byť klasifikované na základe charakteru dát, s ktorými pracujú. Najčastejšie môžme prúdu klasifikovať do kategorických alebo numerických prúdov.
	\item \textit{kompletnosť štatistickej informácie}, mnoho aplikácií reálneho sveta nemá normálne rozdelenie dát, preto je potrebné aplikovať modely a metódy, ktoré sú parametrické alebo semi-parametrické. Ďalej je potrebné spomenúť neparametrické metódy, ktoré pracujú obvykle s posuvným oknom.
	\item \textit{rýchlosť prúdiacich dát}, boli navrhnuté rámce a metódy pre detekciu zmeny na základe zmeny hustoty vznikajúcich dát v vopred používateľom zadefinovanom časovom okne.
	%TODO: Zmeny - pridat dalsie z clanku
\end{itemize}

\section{Detekcia anomálií}
Detekcia anomálií predstavuje proces identifikácie dát, ktoré sa význačne odchyľujú (angl. deviate) od historických vzorov \citep{hodge2004survey}. Anomálie môžu spôsobovať chyby v meraní senzorov, nezvyčajné správanie systému alebo chyba pri prenose dát, či zámerné vytváranie anomálií v používateľmi generovanom obsahu. 
Takže detekcia anomálií má veľa praktického použitia napríklad v aplikáciach, ktoré dohliadajú na kvalitu a kontrolu dát \citep{hill2007real} alebo adaptívne monitorovanie sietí  \citep{hill2010anomaly}. Tieto aplikácie často kladú požiadavku aby boli anomálie detekované v čase ich v vzniku, teda v reálnom čase. Potom metódy pre detekciu anomálií musia byť rýchle vo vykonávaní a mať inkrementálny charakter.\\

V minulosti sa obvykle anomálie detekovali manuálne s pomocou vizualizačných nástrojov, ktoré doménovým expertom pomáhali v tejto úlohe. Manuálne metódy avšak zlyhávajú pri detekcií anomálií v reálnom čase. Výskumníci navrhli niekoľko metód, ktoré majú myšlienku v prístupoch strojového učenia sa a automatizovaného štatistického vyhodnocovania \citep{hill2010anomaly}: \textit{minimálny objem elipsoidu}, \textit{konvexný zvon}, \textit{najbližší sused}, \textit{zhlukovanie}, \textit{klasifikácia neurónovou sieťou}, \textit{klasifikácia strojom podporných vektorov} a \textit{rozhodovacie stromy}. Tieto metódy sú pochopiteľne rýchlejšie než manuálna detekcia, avšak jeden význačný nedostatok, niesú vhodné pre prúdové spracovanie v reálnom čase.
\paragraph{Dátovo riadená metóda} (angl. data-driven), ktorú navrhli \citep{hill2010anomaly}, využíva dátovo riadený jednorozmerný autoregresívny model prúdu dát a predikčný interval (ďalej len PI) vypočítaný z posledných historických dát na identifikáciu anomálií v prúde. Dátovo riadený model časového radu je použitý, pretože je jednoduchší na implementáciu a použitie v porovnaní s ostatnými modelmi časových radov. Tento model tiež poskytuje rýchle a presné prognózy. Dáta sú potom klasifikované ako anomálie na základe toho, či sú spadnú do zvoleného intervalu PI. Metóda teda poskytuje principiálny rámec pre výber hraničného prahu kedy majú byť anomálie klasifikované. Výhoda metódy je, že nevyžaduje žiadne vzorky dát, ktoré sú vopred označkované alebo klasifikované. Je veľmi dobre škálovateľná na veľké objemy dát a vykonáva inkrementálne počítanie tak ako dáta vznikajú.
Metóda pozostáva z nasledujúcich krokov so začiatkom v čase \textit{t}:
\begin{enumerate}
	\item použi model na predikciu o krok vpred (angl. one-step-ahead), ktorý má ako vstup $\displaystyle D^t = \{x_{t-q+1}, ..., x_t\}$ \textit{q} je rôzne meranie \textit{x} v čase \textit{t} a $\displaystyle D^t$ je model predikcie. Tento model je použitý ne predikovanie hodnoty $\displaystyle \overline{x}_{t+1}$ ako očakávaná hodnota v čase \textit{t+1}.
	\item výpočet hornej a spodnej hranice kam by malo spadnúť pozorované meranie s pravdepodobnosťou \textit{p}.
	\item porovnaj pozorovanie v čase \textit{t+1}, či spadá do určeného intervalu. Ak spadne mimo intervalu, objekt je klasifikovaný ako anomália.
	\item 
		\begin{enumerate}
			\item pri stratégii metódy detekcie anomálií a zmiernenia (angl. anomaly detection and mitigation) ADAM, ak je pozorovaný objekt klasifikovaný ako anomália, modifikuj $\displaystyle D^t$ odstránením $\displaystyle x_{t-q+1}$ z konca pozorovaného okna a pridaním $\displaystyle \overline{x}_{t+1}$ na začiatok okna, čím vytvoríme $\displaystyle D^{t+1}$.
			\item pri jednoduchej stratégii detekcie anomálií (angl. anomaly detection) AD, modifikuj $\displaystyle D^t$  odstránením $\displaystyle x_{t-q+1}$ z konca okna a pridaj $\displaystyle x_{t+1}$ na začiatok okna čím vznikne $\displaystyle D^{t+1}$.
		\end{enumerate}
	\item opakuj kroky \textit{1-4}
\end{enumerate}
\paragraph{Metóda dynamických bayesových sietí} (angl. Dynamic Bayesian Networks) \citep{hill2007real} bola vytvorená pre detekciu anomálií v prúdoch zo senzorov, ktoré sú umiestnené v životnom prostredí. Bayesové siete predstavujú acyklický orientovaný graf, zobrazené na obrázku \ref{fig:anomalie-dbn}, v ktorom každý uzol obsahuje pravdepodobnostú informáciu v súvislosti k všetkým možným stavom, v ktorých sa môže premenná nachádzať. Táto informácia spolu s topológiou bayesovej siete, špecifikuje úplné spojenie distribúcie stavu premennej, pričom sada známych premmených môže byť použitá na odvodenie hodnoty neznámych premenných. Dynamické bayesové siete s topológiou, ktorá sa vyvýja v čase, pridáva nové stavové premenné pre lepšiu reprezentáciu stavu systému v aktuálnom čase \textit{t}. Stavové premmné môžeme kategorizovať ako \textit{neznáme}, ktoré predstavujú skutočný stav systému a \textit{merané}, ktoré sú nedonalé merania. Tieto premenné môžu byť naviac diskrétne alebo spojité. Nakoľko sa veľkosť siete zväčšuje s časom, vytváranie záverov použitím celej siete by bolo neefektívne a časovo náročné. Preto boli vyvinuté aproximačné algoritmy ako \textit{Kalmanové filtrovanie} alebo \textit{Rao-Blackwellized časticové filtrovanie}.\\
Hill et al. navrhli v \citep{hill2007real} dve stratégie pre detekovanie anomálií v prúde dát:
\begin{itemize}
	\item \textit{Bayesov dôveryhodný interval} (angl. Bayesian credible interval - BCI), ktorý sleduje viacrozmernú gausovskú distribúciu lineárneho stavu premennej, ktorý korešponduje s neznámym stavom systému a jej meraným náprotivkom.
	\item \textit{Maximálne posteriori meraný status} (angl. Maximum a posteriori measurement status - MAP-ms) používa komplexnejšiu dynamickú bayseovú sieť. Princíp je rovnaký ako pri BCI, pričom MAP-ms metóda je naviac rozšírená o status (napr. anomália áno/nie), ktorý je reprezentovaný distribúciou diskrétnej premennej každého merania senzoru.
\end{itemize}
\myFigure{images/2_anomalie_DBN}{Štruktúra dnamickej bayseovej siete. Vektor $X$ reprezentuje spojitú zložku, neznáme alebo tiež nazývané skryté premenné systému a vektory $M$ predstavujú spojité pozorované premenné v čase $t$.}{anomalie-dbn}{0.65}{h!}\label{fig:anomalie-dbn}


\section{Detekcia trendov}
Detekcia trendov predstavuje kritickú úlohu pre analytikov. Reagovať na vzniknutý trend v čase jeho vzniku môže mať kritické dopady na fungovanie spoločnosti. Preto existuje záujem detekovať trendy v čase ich vzniku a byť schopný adekvátne reagovat príslušnými akciami v reálnom čase. Ak hovoríme o trendoch v obsahu, ktorý je generovaný používateľmi, napríklad na sociálnej sieti, potom sú trendy typicky poháňané udalosťami, ktoré náhle vznikajú a používatelia javia o ne záujem \citep{mathioudakis2010twittermonitor}. Mathioudakis and Koudas navrhli a implementovali metódu na detekciu trendov na sociálnej sieti Twitter\footnote{https://twitter.com/}. Metóda vykonáva detekciu trendov a ich následnú dodatočnú analýzu. Detekcia trendu pozostáva z dvoch krokov:
\begin{enumerate}
	\item \textit{detekcia nárazových kľúčových slov} identifikuje keď sa kľúčove slovo $K$ začne vyskytovať v prúde s neobvykle vysokým podielom v prúde. Napríklad náhly nárast frekvencie kľúčového slova \textit{NBA} môže byť spojený s prebiehajúcim dôležitým zápasom NBA. Pre detekciu nárazových kľúčových slov navrhli \citep{mathioudakis2010twittermonitor} nový algoritmus  \textit{QueueBurst} s nasledujúcimi charakteristikami:
		\begin{enumerate}
			\item \textit{jeden prechod} (angl. one-pass). Keďže ide o prúdové spracovanie, dáta môžu byť prečítané iba raz.
			\item \textit{spracovanie v reálnom čase}. Identifikácia nárazových kľúčových slov je vykonávané tak ako dáta vznikajú.
			\item \textit{odolnosť voči falošným nárazovým kľúčovým slovám}. Niekedy sa stane, že kľúčové slovo začne nárazovo prúdiť, ale nemusí predstavovať prúd, môže sa vyskytnúť zhodou okolností.
			\item \textit{odolnosť voči spam-u}. Existuje veľa automatických botov a používateľov, ktorí generujú spamujúce správy. Spam by mohol značne znížiť presnosť detekcie trendu.
		\end{enumerate}		
	\item \textit{zoskupovanie nárazových kľúrových slov} po tom čo algoritmus \textit{QueueBurst} identifikuje $\displaystyle K_t$ kľúčových slov pre každý časový moment $t$, sú kľúčové slová $\displaystyle k \in K_t$ periodicky zoskupované do nesúvislých (angl. disjoint) podmnožín $\displaystyle K_t^i \in K_t$. Potom identifikovaný trend predstavuje podmnožina $\displaystyle K_t^i$. Zoskupovanie vykonáva algoritmus \textit{GroupBurst}, ktorý posuďuje spoločný výskyt v posledných správach. Algoritmus je realizovaný lačnou stratégiou.
	\item Posledným krokom je analýza identifikovaného trendu $\displaystyle K_t^i$. Prvým krokom je identifikovať ďalšie kľúčové slová, ktoré sa spájajú s trendom $\displaystyle K_t^i$. Toto je dosiahnuté algoritmami na extrakciu kontextu, ktoré sú spustené na nedávnej histórií správ. Algoritmus vráti kľúčové slová, ktoré najviac korelujú s identifikovaným trendom $\displaystyle K_t^i$. Navyše, trendy na sociálnej sieti často pozostávajú z komentárov na aktuálne správy a novinky vo svete (napr. nytimes.com). Preto má zmysel ďalej extrahovať aj príslušné hypertextové odkazy a prideliť ich k trendu $\displaystyle K_t^i$. Posledný krok tejto metódy je zobrazenie priebehu identifikovaného trendu pomocou vizualizácie pre používateľa.
\end{enumerate}
%TODO: nakresliť obrázok z článku twittermonitor
%TODO: pridať ďalšie dve metódy z článku

\section{Rozpoznanie pocitu a nálady z používateľom generovaného obsahu}
Analýza pocitu alebo nálady (angl. sentiment analysis) može byť chápaný ako problém klasifikácie. Úlohoou je to klasifikovať správy (najčastejšie v kontexte sociálnych sietí) do dvoch kategórií na zákalde ich pozitívnych alebo negatívnych dojmov. Ak by sme pracovali s dátami zo sociálnej siete Twitter, je možné použiť na označkovanie správ. pomerne dobre, extrahovaním emotikonov, ktoré vyjadrujú pocity používateľa \citep{bifet2010sentiment}. \\
Bifet and Frank publikovali v práci \citep{bifet2010sentiment} tri metódy a ich overenie na rozpoznanie nálady a pocitov z používateľom generovaného obsahu na sociálnej sieti Twitter. Experimentovali s tromi inkrementálnymi metódami, ktoré sú vhodné na spracovanie prúdu dát.

\paragraph{Multinomiálny Naive Bayes} je klasifikátor najčastejšie používaný na klasifikáciu dokumentov, ktorý obvykle poskytuje dobré výsledky aj čo sa týka presnosti výsledku aj rýchlosti. Túto metódu je jednoduché aplikovať v kontexte prúdu dát \citep{bifet2010sentiment}. Multinomiálny naivný Bayes sa pozerá na dokument ako na zhluk slov. Pre každú triedu $c$, $P(w|c)$, pravdepodobnosť, že slovo $w$ patrí do tejto triedy je odhadovaná z trénovacích dát jednoducho vypočítaním relatívnej početnosti každého slova v trénovacej sade pre danú triedu. Klasifikátor potrebuje naviac nepodmienenú pravdepodobnosť $P(c)$. Za predpokladu, že $\displaystyle n_{wd}$ je počet výskytov slova $w$ v dokumente $d$, pravdepodobnosť triedy $c$ z testovacieho dokumentu je nasledovaná: \newline
\begin{align*}
P(c|d) = \frac{P(c)\prod _{w \in d} P(w|c)^{n_{wd}}} {P(d)}
\end{align*}
Kde $P(d)$ je normalizačný faktor. Aby sme sa vyhli problému kedy sa trieda nevyskytuje v datasete ani jeden krát, je bežné použitie Laplacovej korekcie a nahradenie nulových početností jednotkou, resp. inicializovať početnosť každej triedy na 1 namiesto 0.

\paragraph{Stochastický gradientný zostup} (angl. Stochastic Gradient Descent, SGD). Bifet and Frank v ich práci použili implementáciu tzv. vanilla stochastický gradientný zostup s pevnou rýchlosťou učenia, optimalizujúc stratu s $L_2$ penalizáciou. $L_2$ penalizácia je často používaná pri podporných vektorových strojoch (angl. support vektor machines). Lineárny stroj, ktorý je často aplikovaný na problémy klasifikácie dokumentov, optimalizujeme funkciu straty nasledovne:
\begin{align*}
\frac{\lambda }{2}\left \| w \right \|^{2}+\sum [1-(yxw + b)]_{+}
\end{align*}
kde $w$ je váhovaný vektor, $b$ je sklon, $\lambda$ regulačný parameter a označenie triedy $y$ je z intervalu $\{+1, -1\}$.

\paragraph{Hoeffdingov strom} (angl. Hoeffding tree) je najznámejšia implementácia rozhodovacích stromov v použití prúdového spracovania. Hoeffdingov algoritmus implementuje stratégiu pred-prerezávania, ktorá je založená na Hoeffdingovom ohraničení. Toto umožňuje inkrementálne budovanie rozhodovacieho stromu. Uzol stromu je rozvinutý hneď ako obsahuje dostatočne silnú štatistickú informáciu. Existujú ďalšie sofistikované implementácie Hoeffdingových stromov, ktoré implementujú rýchlejšie a efektívnejšie algoritmy, napr. VFDT - Very Fast Decision Trees \citep{domingos2000mining} alebo CVFDT - Concept adapting Very Fast Decision Trees \citep{hulten2001mining}

\section{ETL proces a kvalita dát}
%TODO: stručne


\section{Zhodnotenie}
TODO: zhodnotenie analytických uloh, aké sú problémy atd.