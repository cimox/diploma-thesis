\chapter{Existujúce nástroje pre analýzu prúdu udalostí}
\label{Existujúce nástroje pre analýzu prúdu udalostí} 

\section{MOA}
Masívny online analyzátor (z angl. Massive Online Analysis - MOA, ďalej len MOA) je softvérové prostredie pre implementáciu algoritmov a vykonávanie experimentov pre online učenie sa z vyvýjajúcich sa prúdov dát \citep{DBLP:journals/jmlr/BifetHKP10}. MOA pozostáva z kolekcie offline a online metód a tiež nástrojov pre evaluáciu týchto metód. MOA implementuje metódy a algoritmy pre klasifikáciu, zhlukovanie prúdu, detekciu inštancií, ktoré sa vymykajú prahovým hodnotám a tiež odporúčacie systémy. Presnejšie MOA implementuje napríklad nasledujúce: stupňovanie (angl. boosting), vrecovanie (angl. bagging) a Hoeffdingove stromy, všetky metódy s a bez Naive Bayes klasifikátorom na listoch. MOA podporuje obojsmernú interakciu s nástrojom WEKA, ktorý je detailne opísaný v nasledujúcej kapitole. \\

MOA je implementovaná v programovacom jazyku Java. Za hlavný benefit implementácie v Jave považujú autori jej platformová nezávislosť. MOA obsahuje tiež generár prúdu dát, vie dobre modelovať concept drift. V aplikácií je možné definovať pravdepodobnosť, že inštancia prúdu patrí do nového concept drift-u. Sú dostupné nasledovné generátory prúdu \citep{DBLP:journals/jmlr/BifetHKP10}: \textit{Random Tree Generator, SEA Concepts Generator, STAGGER Concepts Generator, Rotating Hyperplane, Random RBF Generator, LED Generator, Waveform Generator, and Function Generator}.
\myFigure{images/3_moa}{Hlavná obrazovka GUI nástroja MOA.}{moa}{0.45}{h!}\label{fig:moa}

\section{WEKA}
The Wakaito Enviroment for Knowledge Analysis (ďalej len Weka) vznikol s jednoduchým cieľom poskytunúť výskumníkom unifikovanú platformu pre prístup k state-of-the-art technikám strojového učenia sa \citep{hall2009weka}. Weka vznikla na University of Waikato na Novom Zélande v roku 1992, pričom je aktívne vyvýjaná posledných 16 rokov. Weka poskytuje kolekciu algoritmov strojového učenia sa pre úlohy dolovania v dátach. Algoritmy môžu byť priamo aplikované na datasety prostredníctvom aplikácie alebo pouťžité vo vlastných aplikáciach volaním Java kódu. Weka obsahuje tieť nástroje na predspracovanie dát, klasifikáciu, regresiu, zhlukovanie, asociačné pravidlá a vizualizáciu. Nástroj je tieť vhodný pre návrhovanie a vývoj nových schém pre strojové učenia sa v kontexte dolovania dát. Zaujímavosťou je tiež, že Weka je nelietajúci vták, ktorý žije len na ostrove Nového Zélandu. \\

Vieľom nástroja je poskytnúť pracovný nástroj pre výskumníkov. Poskytuje napríklad (nástroj ich obsahuje omnoho viac, vymenované sú len vybrané) tieto algoritmy určené pre klasifikáciu dát:
\begin{itemize}
	\item \textit{Bayesová logistická regresia} (angl. Bayesian logistic regression), pre kategorizáciu textu s Gausovským a Laplacovým apriori.
	\item \textit{Najlepší prvý rozhodovací strom} (angl. Best-first decision tree), konštrukcia rozhodovacieho stromu so stratégiou najlepší prvý.
	\item \textit{Hybridná rozhodovacia tabuľka a naivný Bayes} (angl. Decision table naive Bayes hybrid) hybridný klasifikátor, ktorý kombinuje rozhodovacie tabuľky a metódu Naivný Bayes.
	\item \textit{Funkčné stromy} sú rozhodovacie stromy s lomeným rozdelením a lineárnymi funkciami v listoch.
\end{itemize}
Weka poskytuje tiež nástroje pre predspracovanie dát, zoznam niektorých filtrov (vymenované sú len vybrané základné filtre):
\begin{itemize}
	\item \textit{Pridanie klasifikátora}, pridá predikcie klasifikátora do datasetu.
	\item \textit{Pridanie ID} ako nového atribútu pre každý záznam datasetu.
	\item \textit{Pridanie hodnoty} chýbajúcim hodnotám z poskytnutého zoznamu.
	\item \textit{Preskupenie atribútov} preusporiadanie poradia atribútov.
	\item \textit{Numerické hodnoty na nominálne}, konverzia numerických hodnôt na nominálne.
\end{itemize}
\myFigure{images/3_weka}{Hlavná obrazovka GUI nástroja WEKA.}{weka}{0.5}{h!}\label{fig:weka}

\newpage
\section{RapidMiner Streams-Plugin}
Streams plugin poskytuje operátory RapidMiner-u pre základné budovanie blokov Streams API použítím obaľovača (angl. wrapper) na priame použitie implementácie, ktorú poskytuje Streams balík. Operátory Streams Plugin-u sú automaticky vytvorené pomocou procesora a použitím knižnice RapidMiner Beans \citep{bockermann2012processing}. Architektúra Streams Plugin-u je postavená na Streams API, ktoré bolo navrhnuté v práci Bockermanna a Bloma.
\myFigure{images/3_rapidminer}{Architektúra RapidMiner Stream Plugin-u a ďalších potrebných častí.}{rapidminer}{0.65}{h!}\label{fig:rapidminer}

\section{StreamBase}
StreamBase\footnote{http://www.streambase.com/} je platforma pre spracovanie udalostí, ktorá poskytuje vysoko-výkonný softvér pre budovanie a nasadanie systémov, ktoré analyzujú a reagujú (napr. akciami) na prúdiace dáta v reálnom čase. StreamBase poskytuje prostredie pre svižný vývoj, server pre spracovanie udalostí s nízkou odozvou a vysokou priepustnosťou a zároveň integráciu do podnikových nástrojov, napríklad pre spracovanie historických údajov. Server analyzuje prúdiace dáta a poskytuje výsledky a odpovede v reálnom čase s extrémne nízkou odozvou. Toto je dosiahnuté maximalizáciou využitia hlavnej pamäte a ostatných prostriedkov servera, zatiaľ čo sa eliminujú závislosti na ostatné aplikácie. Integrované vývojové prostredie - StreamBase Studio umožňuje programátorom jednoducho a rýchlo vytvoriť, testovat a debugovať StreamSQL aplikácie použitím grafického modelu toku vykonávania. StreamBase aplikácie sú potom skompilované a nasadené za behu servera. \\

StreamSQL je dopytovací jazyk, ktorý rozširuje štandard SQL. StreamSQL umožňuje spracovanie prúdov v reálnom čase a dopytovanie sa do nich. Základná myšlienka jazyka SQL je možnosť dopytovať sa do uložených statických kolekcií dát, StreamSQL umožňuje to isté, ale do prúdov dát. Teda, StreamSQL musí zvládnuť spracovať kontinuálny prúd udalostí a časovo orientované záznamy. StreamSQL zachováva schopnosti jazyka SQL zatiaľ čo pridáva nové možnosti ako napríklad: bohatý systém posuvných okien, možnosť miešania prúdiacich dát a statických dát a tiež možnosť pridať vlastnú logiku vo forme analytických funkcií. \\

StreamBase EventFlow je jazyk pre prúdove spracovanie vo forme tokov a operátorov ako grafických elementov. Používateľ má možnosť spájať tieto grafické elementy a vytvárať tak jednoducho topológiu pre prúdové spracovanie bez nutnosti programovania. EventFlow integruje všetky možnosti StreamSQL. \\

Použitie StreamBase sa výborne hodí pre štrukturované aplikácie "reálneho času", ktoré majú za cieľ rýchle spracovanie spolu s rýchlym prototypovaním a nasadením nových funkcionalít.
\myFigure{images/3_streambase}{Vývojové prostredie nástroja StreamBase.}{streambase}{0.55}{h!}\label{fig:streambase}

%\section{InforSphere Streams (IBM)}
\section{Spark}
Spark je klastrový výpočtový systém. Kombinuje spracovanie uložených dát v dávkovom móde so spracovaním prúdu údajov v reálnom čase \citep{cimerman2015prudy}. Cieľom Spark-u je poskytnúť rýchlu výpočtovú platformu pre analýzu dát. Spark poskytuje všeobecný model výkonávania ľubovolných dopytov, ktoré sú výkonávané v hlavnej pamäti (pokiaľ ide o prúdové spracovanie). Tento model je nazvaný \textit{Pružný distribuovaný dataset}, skr. RDD (angl. Resilient Distributed Dataset), čo je dátova abstrakcia distribuovanej pamäti. Keďže výpočet beží v hlavnej pamäti (pri prúdovom spracovaní), nie je potrebné vykonávať zápisy na disk, vďaka čomu môže byť dosiahnuté spracovanie v reálnom čase. Výpočet prebieha vo veľkom klastri uzlov s dosiahnutím odolnosti voči chybám za použitia RDD. RDD  sídli v hlavnej pamäti, ale môže byť periodicky ukladaný na disk. Vďaka distribuovanej povahe RDD môže byť stratená časť RDD obnovená z pamäti iného uzla. Samotné prúdové spracovanie nie je vykonávané správa po správe (angl. message by message), ale v mikro dávkach, ktoré môžu byť automaticky paralelne distrubované v strapci.\\
Spark Streaming\footnote{http://spark.apache.org/streaming/} alebo tiež, prúdové spracovanie, si v poslednej dobe vyžiadal špeciálnu pozornosť od tvorcom programovacieho rámca. Reagujú tým na vysoký dopyt odbornej verejnosti po prúdovom spracovaní dát, ktoré chýbalo v Spark-u. Spark Streaming poskytuje integrované rozhranie API pre rôzne programovacie jazyky, pričom v budúcnosti je snaha toto rozhranie úplne integrovať s dávakovým spracovaním, aby mohli vývojári používať rovnaké dátové typy pre rôzne typy úloh. Poskytuje tiež aspoň raz (angl. at least once) schému doručenia správ a zaručuje tak odolnosť voči chybám a prípadnej strate správy. Prúdové spracovanie v Spark-u je jednoduché integrovať spolu s dávkovým spracovaním, ktoré poskytuje, či použiť spolu s knižnicou pre strojové učenie sa. 

%Apache Flink
%Apache Storm



















